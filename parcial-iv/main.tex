\documentclass[a4paper]{article} 
\input{head}
\newcommand{\pow}[2]{#1^{#2}}
\newcommand{\supra}[1]{\textsuperscript{#1}}
\begin{document}

%-------------------------------
%	TITLE SECTION
%-------------------------------

\fancyhead[C]{}
\hrule \medskip % Upper rule
\begin{minipage}{0.35\textwidth} 
\raggedright
\footnotesize
Fausto David Hernández Jasso \hfill\\   
317000928 \hfill\\
fausto.david.hernandez.jasso@ciencias.unam.mx
\end{minipage}
\begin{minipage}{0.4\textwidth} 
\centering 
\large 
Cálculo Diferencial e Integral II\\ 
\normalsize 
Tarea-Examen Parcial 4\\ 
\end{minipage}
\begin{minipage}{0.24\textwidth} 
\raggedleft
\today\hfill\\
\end{minipage}
\medskip\hrule 
\bigskip
\section{Ejercicios}
\subsection{Ejercicio 1}
\noindent
Demostrar usando los polinomios de Taylor y el residuo que si \(f''(a)\) existe, entonces
\begin{align*}
    f''(a) &= \lim_{h \to 0} \frac{f(a + h) + f(a - h) -2f(a)}{h^2}
\end{align*}
\subsubsection*{Solución}
\subsection{Ejercicio 2}
\noindent
Encuentra el valor de \(e\) hasta el primer decimal utilizando polinomios de Taylor.
\subsubsection*{Solución}
\subsection{Ejercicio 3}
\noindent
Demuestra que el polinomio de Taylor de \(f(x) = \cos{x^2}\) de grado \(4n\) en 0 es
\[
    1 - \frac{x^4}{2!} + \frac{x^8}{4!} - \dotsc + (-1)^{n}\frac{x^{4n}}{(2n)!}
\]
\subsubsection*{Solución}
\begin{proof}
    Primero calcularemos el polinomio de Taylor de \(\cos{x}\).
    Por definición tenemos que el polinomio de Taylor de grado \(n\) de una función \(f\), en \(0\)
    es
    \begin{align*}
        P_{n, 0, f} = \sum_{k = 0}^{n} \frac{f^{(k)}(0)}{k!}x^{k}
    \end{align*}
    donde \(f^{(k)}\) es la \(k-\)ésima derivada de la función \(f\). En éste caso 
    \(f(x) = \cos{x}\), como \(\cos{x}\) tiene derivadas en todos los órdenes para cada 
    \(x \in \mathbb{R}\), es decir, \(\cos{x}\) es de clase \(C^{\infty}\) en \(\mathbb{R}\).
    \newline 
    Notemos lo siguiente:
    \begin{align*}
        f'(x) &= -\sin{x} \\
        f''(x) &= -\cos{x} \\
        f^{(3)}(x) &= \sin{x} \\
        f^{(4)}(x) &= \cos{x} \\
        f^{(5)}(x) &= -\sin{x} \\
        f^{(6)}(x) &= -\cos{x} \\
        f^{(7)}(x) &= \sin{x} \\
        f^{(8)}(x) &= \cos{x}
    \end{align*}
    En general tenemos que
    \begin{align*}
        f^{(4n)}(x) &= \cos{x} \\
        f^{(4n + 1)}(x) &= -\sin{x} \\
        f^{(4n + 2)}(x) &= -\cos{x} \\
        f^{(4n + 3)}(x) &= \sin{x} \\
    \end{align*}
    Observamos que
    \begin{align*}
        f^{(4n)}(0) &= 1 \\
        f^{(4n + 1)}(0) &= 0 \\
        f^{(4n + 2)}(0) &= -1 \\
        f^{(4n + 3)}(0) &= 0
    \end{align*}
    con \(n \in \mathbb{N} \cup \{0\}\).
    \newline
    Consecuentemente todos los términos impares nos van a quedar siempre \(0\). Para 
    facilitar el desarrollo del polinomio de Taylor eligiremos un grado par que son 
    los términos que no son \(0\). Por lo tanto,
    \begin{align*}
        P_{2n, 0, \cos{x}} &= \sum_{k=0}^{2n} \frac{f^{(k)}(0)}{k!}x^{k} \\
        &= f(0) + \frac{f'(0)}x + \frac{f''(0)}{2!}x^{2} + \frac{f^{(3)}(0)}{3!}x^{3} + \dotsc + \frac{f^{(2n - 1)}(0)}{(2n - 1)!}x^{2n - 1} +\frac{f^{(2n)}(0)}{(2n)!}x^{2n} \\
        &= 1 + 0 - \frac{x^2}{2!} + 0 + \frac{x^4}{4!} + 0 - \frac{x^6}{6!} + \dotsc + \frac{(-1)^{n}x^{2n}}{(2n!)} \\
        &= 1 - \frac{x^2}{2!} + \frac{x^4}{4!} - \frac{x^6}{6!} + \dotsc + \frac{(-1)^{n}x^{2n}}{(2n!)} \\
    \end{align*}
    Por definición tenemos que 
    \begin{align*}
        \cos_{x} &= P_{2n, 0, \cos{x}}(x) + R_{2n, 0, \cos{x}}(x) \\
    \end{align*}
    Sabemos que si \(f\) es una función tal que \(f'(a), \dotsc, f^{(n)}(a)\) existen. Si 
    \(P\) es un polinomio en \((x-a)\) de grado menor o igual a \(n\), igual a \(f\) 
    hasta el orden \(n\) en \(a\), entonces \(P = P_{n, a, f}\).
    Como \(\cos{x}\) tiene derivadas en todos los órdenes para cada 
    \(x \in \mathbb{R}\), es decir, \(\cos{x}\) es de clase \(C^{\infty}\) en \(\mathbb{R}\), 
    entonces \(\cos^{(0)}{x}, \cos^{(1)}{x}, \dotsc, \cos^{(2n)}{x}\) existen, tenemos que 
    \begin{align*}
        \lim_{x \to 0} \frac{\cos{x} - P_{2n, 0, \cos{x}}(x)}{x^{2n}} &= \lim_{x \to 0} \frac{R_{2n, 0, \cos{x}}(x)}{x^{2n}}\\
        &= 0
    \end{align*}
    Ya que \(P_{2n, 0, \cos{x}}\) es el polinomio de Taylor de \(\cos{x}\).
    \newline
    Lo anterior implica que
    \begin{align*}
        \lim_{x \to 0} \frac{R_{2n, 0, \cos{x}}(x^2)}{(x^2)^{2n}} &= 0
    \end{align*}
    Esto por definición es
    \begin{align*}
        \lim_{x \to 0} \frac{f(x^2) - P_{2n, 0, \cos{x}}(x^2)}{x^{4n}} &= 
        \lim_{x \to 0} \frac{\cos{x^2} - P_{2n, 0, \cos{x}}(x^2)}{x^{4n}} \\
    \end{align*}
    Sabemos que si \(f\) es una función tal que \(f'(a), \dotsc, f^{(n)}(a)\) existen. Si 
    \(P\) es un polinomio en \((x-a)\) de grado menor o igual a \(n\), igual a \(f\) 
    hasta el orden \(n\) en \(a\), entonces \(P = P_{n, a, f}\).
    Por lo tanto 
    \begin{align*}
        P_{4n, 0, \cos{x^2}}(x) = P_{2n, 0, \cos{x}}(x^2)
    \end{align*}
    Consecuentemente 
    \begin{align*}
        P_{4n, 0, \cos{x^2}}(x) = &= 1 - \frac{x^4}{2!} + \frac{x^8}{4!} - \frac{x^{12}}{6!} + \dotsc + \frac{(-1)^{n}x^{4n}}{(2n!)} \\
    \end{align*}
\end{proof}
\subsection{Ejercicio 4}
\noindent
Supongamos que \(f\) es dos veces derivable en \((0, \infty)\) y que \(|f(x)| \leq M_{0}\) para 
todos  los \(x > 0\) mientras que \(|f''(x)| \leq M_{2}\). Demuestra que
\begin{itemize}
    \item Para todo \(x > 0\) y todo \(h > 0\) se cumple
    \[
        |f'(x)| \leq \frac{2}{h}M_{0} + \frac{h}{2}M_{2}
    \]
    \item Concluir que \(|f'(x)| \leq 2 \sqrt{M_{0}M_{2}}\)
\end{itemize}
\subsubsection*{Solución}
\subsection{Ejercicio 5}
\noindent
Aproximar los siguientes números con sumas cuyo error sea menor al indicado
\begin{itemize}
    \item \(\sin{1}\), error \(< 10^{-9}\).
    \item \(e^{2}\), error \(< 10^{-5}\)
\end{itemize}
\subsubsection*{Solución}
Veamos \(\sin{1}\).
\newline 
Consideremos \(f : \mathbb{R} \to \mathbb{R}\) definida como
\[
    f(x) = \sin{x}
\]
Por definición sabemos que 
\[
    f(x) = P_{n, 0, f}(x) + R_{n, 0, f}(x)
\]
En particular, 
\begin{align*}
    \sin{1} &= f(1) \\
    &= P_{n, 0, f}(1) + R_{n, 0, f}(1) \\
\end{align*}
Lo que buscamos es que 
\[
    | R_{n, 0, f}(x) | < 10^{-9}
\]
Recordando el polinomio de Taylor de grado \(2n + 1\) en \(0\) tenemos que
\[
    P_{2n + 1, 0, f}(x) = x - \frac{x^3}{3!} + \frac{x^5}{5!} - \frac{x^7}{7!} + \dotsc + \frac{(-1)^{n}}{(2n + 1)!}x^{2n + 1}
\]
Es decir el término \(k-\)ésimo del polinomio de Taylor de grado \(2n + 1\) en \(0\) tenemos que
\[
    \frac{(-1)^{k}}{(2k + 1)!}x^{2k + 1}
\]
Como \(\sin{x}\) tiene derivadas en todos los órdenes para cada 
\(x \in \mathbb{R}\), es decir, \(\cos{x}\) es de clase \(C^{\infty}\) en \(\mathbb{R}\).
en particular existe \(\sin^{(n+1)}{x}\), además sea \(y = 1\) y \(a = 0\), se tiene que 
\(y \leq a\), notemos además lo siguiente:
\begin{align*}
    f'(x) &= \cos{x} \\
    f''(x) &= -\sin{x} \\
    f^{(3)}(x) &= -\cos{x} \\
    f^{(4)}(x) &= \sin{x} \\
    f^{(5)}(x) &=  \cos{x}  \\
    f^{(6)}(x) &= -\sin{x}  \\
    f^{(7)}(x) &= -\cos{x} \\
    f^{(8)}(x) &=  \sin{x} 
\end{align*}
En general tenemos que
\begin{align*}
    f^{(4n)}(x) &= \sin{x} \\
    f^{(4n + 1)}(x) &= \cos{x} \\
    f^{(4n + 2)}(x) &= -\sin{x} \\
    f^{(4n + 3)}(x) &= -\cos{x} \\
\end{align*}
Así tenemos que \(| f^{(n+1)}(t) | \leq 1\) para toda \(t \in [a, y]\) por un corolario del 
Teorema de Taylor tenemos que 
\begin{align*}
    | R_{n, 0, f}(t) | &\leq \frac{1}{(n + 1)!}|1|^{n} = \frac{1}{(n + 1)!}
\end{align*}
para cada \(t \in [a, y]\).
\newline 
Entonces lo que buscamos es que 
\[
    \frac{1}{(n + 1)!} < 10^{-9}
\]
La \(n\) que cumple con lo solicitado es \(n = 11\), ya que 
\begin{align*}
    \frac{1}{(11 + 1)!} = \frac{1}{(12)!} = 1{.}6059 \times 10^{-10} & < 10^{-9}
\end{align*}
Así 
\[
    \sin{1} \approx P_{11, 0, f}(1) = 1 - \frac{1}{3!} + \frac{1}{5!} - \frac{1}{7!} + \frac{1}{9!} - \frac{1}{11!}
    = 0{.}841470984808
\]
\end{document}