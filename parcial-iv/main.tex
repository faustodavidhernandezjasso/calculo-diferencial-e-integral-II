\documentclass[a4paper]{article} 
\addtolength{\hoffset}{-2.25cm}
\addtolength{\textwidth}{4.5cm}
\addtolength{\voffset}{-3.25cm}
\addtolength{\textheight}{5cm}
\setlength{\parskip}{0pt}
\setlength{\parindent}{0in}

%----------------------------------------------------------------------------------------
%	PACKAGES AND OTHER DOCUMENT CONFIGURATIONS
%----------------------------------------------------------------------------------------
\usepackage{mathtools} % Package for using math tools
\usepackage[inline]{enumitem} % Enumerate environment
\usepackage{blindtext} % Package to generate dummy text
\usepackage{charter} % Use the Charter font
\usepackage[utf8]{inputenc} % Use UTF-8 encoding
\usepackage{microtype} % Slightly tweak font spacing for aesthetics
\usepackage[spanish]{babel} % Language hyphenation and typographical rules
\usepackage{amsthm, amsmath, amssymb} % Mathematical typesetting
\usepackage{float} % Improved interface for floating objects
\usepackage[final, colorlinks = true, 
            linkcolor = black, 
            citecolor = black]{hyperref} % For hyperlinks in the PDF
\usepackage{graphicx, multicol} % Enhanced support for graphics
\usepackage{xcolor} % Driver-independent color extensions
\usepackage{marvosym, wasysym} % More symbols
\usepackage{rotating} % Rotation tools
\usepackage{censor} % Facilities for controlling restricted text
\usepackage{listings, style/lstlisting} % Environment for non-formatted code, !uses style file!
\usepackage{pseudocode} % Environment for specifying algorithms in a natural way
\usepackage{style/avm} % Environment for f-structures, !uses style file!
\usepackage{booktabs} % Enhances quality of tables
\usepackage{tikz-qtree} % Easy tree drawing tool
\tikzset{every tree node/.style={align=center,anchor=north},
         level distance=2cm} % Configuration for q-trees
\usepackage{style/btree} % Configuration for b-trees and b+-trees, !uses style file!
\usepackage[backend=biber,style=numeric,
            sorting=nyt]{biblatex} % Complete reimplementation of bibliographic facilities
\addbibresource{ecl.bib}
\usepackage{csquotes} % Context sensitive quotation facilities
\usepackage[yyyymmdd]{datetime} % Uses YEAR-MONTH-DAY format for dates
\renewcommand{\dateseparator}{-} % Sets dateseparator to '-'
\usepackage{fancyhdr} % Headers and footers
\usepackage{physics}
\pagestyle{fancy} % All pages have headers and footers
\fancyhead{}\renewcommand{\headrulewidth}{0pt} % Blank out the default header
\fancyfoot[L]{} % Custom footer text
\fancyfoot[C]{} % Custom footer text
\fancyfoot[R]{\thepage} % Custom footer text
\newcommand{\note}[1]{\marginpar{\scriptsize \textcolor{red}{#1}}} % Enables comments in red on margin


%----------------------------------------------------------------------------------------

\newcommand{\pow}[2]{#1^{#2}}
\newcommand{\supra}[1]{\textsuperscript{#1}}
\begin{document}

%-------------------------------
%	TITLE SECTION
%-------------------------------

\fancyhead[C]{}
\hrule \medskip % Upper rule
\begin{minipage}{0.35\textwidth} 
\raggedright
\footnotesize
Fausto David Hernández Jasso \hfill\\   
317000928 \hfill\\
fausto.david.hernandez.jasso@ciencias.unam.mx
\end{minipage}
\begin{minipage}{0.4\textwidth} 
\centering 
\large 
Cálculo Diferencial e Integral II\\ 
\normalsize 
Tarea-Examen Parcial 4\\ 
\end{minipage}
\begin{minipage}{0.24\textwidth} 
\raggedleft
\today\hfill\\
\end{minipage}
\medskip\hrule 
\bigskip
\section{Ejercicios}
\subsection{Ejercicio 1}
\noindent
Demostrar usando los polinomios de Taylor y el residuo que si \(f''(a)\) existe, entonces
\begin{align*}
    f''(a) &= \lim_{h \to 0} \frac{f(a + h) + f(a - h) -2f(a)}{h^2}
\end{align*}
\subsubsection*{Solución}
\begin{proof}
    Como \(f'(a)\) y \(f''(a)\) entonces el polinomio de Taylor de grado \(2\) de la función 
    \(f\) en \(a\) es:
    \begin{align*}
        P_{2, a, f}(x) = f(a) + f'(a)(x - a) + \frac{1}{2}f''(a)(x - a)^2
    \end{align*}
    Sabemos que por definición tenemos que
    \begin{align*}
        f(x) &= P_{2, a, f}(x) + R_{2, a, f}(x)
    \end{align*}
    Evaluando en \(x = a + h\) tenemos que
    \begin{align*}
        f(a + h) &= P_{2, a, f}(a + h) + R_{2, a, f}(a + h) \\
        &= f(a) + f'(a)(a + h - a) + \frac{1}{2}f''(a)(a + h - a)^2 + R_{2, a, f}(a + h) \\
        &= f(a) + f'(a)(h) + \frac{1}{2}f''(a)(h)^2 + R_{2, a, f}(a + h) \\
        &= f(a) + f'(a)h + \frac{1}{2}f''(a)h^2 + R_{2, a, f}(a + h) \\
    \end{align*}
    Evaluando en \(x = a - h\) tenemos que
    \begin{align*}
        f(a - h) &= P_{2, a, f}(a - h) + R_{2, a, f}(a - h) \\
        &= f(a) + f'(a)(-h) + \frac{1}{2}f''(a)(-h)^2 + R_{2, a, f}(a - h) \\
        &= f(a) - f'(a)(h) + \frac{1}{2}f''(a)h^2 + R_{2, a, f}(a - h) \\
        &= f(a) - f'(a)h + \frac{1}{2}f''(a)h^2 + R_{2, a, f}(a - h) \\
    \end{align*}
    Calcularemos \(f(a + h) + f(a - h)\)
    \begin{align*}
        f(a + h) + f(a - h) &= f(a) + f'(a)h + \frac{1}{2}f''(a)h^2 + R_{2, a, f}(a + h) + f(a) - f'(a)h + \frac{1}{2}f''(a)h^2 + R_{2, a, f}(a - h) \\
        &= f(a) + f(a) + f'(a)h - f'(a)h + \frac{1}{2}f''(a)h^2 + \frac{1}{2}f''(a)h^2 + R_{2, a, f}(a + h) + R_{2, a, f}(a - h) \\
        &= 2f(a) + f''(a)h^2 + R_{2, a, f}(a + h) + R_{2, a, f}(a - h) \\
    \end{align*}
    Calcularemos 
    \begin{align*}
        \lim_{h \to 0} \frac{f(a + h) + f(a - h) -2f(a)}{h^2} &= \lim_{h \to 0} \frac{f(a + h) + f(a - h) -2f(a)}{h^2} \\
        &= \lim_{h \to 0} \frac{2f(a) + f''(a)h^2 + R_{2, a, f}(a + h) + R_{2, a, f}(a - h) - 2f(a)}{h^2} \\
        &= \lim_{h \to 0} \frac{2f(a) - 2f(a) + f''(a)h^2 + R_{2, a, f}(a + h) + R_{2, a, f}(a - h)}{h^2} \\
        &= \lim_{h \to 0} \frac{f''(a)h^2 + R_{2, a, f}(a + h) + R_{2, a, f}(a - h)}{h^2} \\
        &= \lim_{h \to 0} \frac{f''(a)h^2}{h^2} + \lim_{h \to 0} \frac{R_{2, a, f}(a + h)}{h^2} + \lim_{h \to 0} \frac{R_{2, a, f}(a - h)}{h^2}  \\
        &= \lim_{h \to 0} f''(a) + \lim_{h \to 0} \frac{R_{2, a, f}(a + h)}{h^2} + \lim_{h \to 0} \frac{R_{2, a, f}(a - h)}{h^2}  \\
        &= f''(a) + \lim_{h \to 0} \frac{R_{2, a, f}(a + h)}{h^2} + \lim_{h \to 0} \frac{R_{2, a, f}(a - h)}{h^2}  \\
    \end{align*}
    Veamos \(\displaystyle\lim_{h \to 0} \frac{R_{2, a, f}(a + h)}{h^2}\)
    \newline 
    Haciendo \(x = a + h\) lo anterior es equivalente a \(h = x - a\)
    \begin{align*}
        \lim_{h \to 0} \frac{R_{2, a, f}(x)}{(x - a)^2} 
    \end{align*}
    Recordemos lo siguiente:
    \begin{align*}
        \lim_{x \to a} \frac{R_{2, a, f}(x)}{(x - a)^2} &=  0
    \end{align*}
    Entonces 
    \begin{align*}
        \lim_{h \to 0} \frac{R_{2, a, f}(a + h)}{h^2} &= 0
    \end{align*}
    Veamos \(\displaystyle\lim_{h \to 0} \frac{R_{2, a, f}(a - h)}{h^2}\)
    \newline 
    Haciendo \(x = a - h\) lo anterior es equivalente a \(h = a - x\)
    \begin{align*}
        \lim_{h \to 0} \frac{R_{2, a, f}(x)}{(a - x)^2} 
    \end{align*}
    Notemos que \((a - x)^2 = a^2 - 2ax + x^2 = (x - a)^2\), entonces 
    \begin{align*}
        \lim_{h \to 0} \frac{R_{2, a, f}(x)}{(a - x)^2} &= \lim_{h \to 0} \frac{R_{2, a, f}(x)}{(x - a)^2} 
    \end{align*}
    Recordemos lo siguiente:
    \begin{align*}
        \lim_{x \to a} \frac{R_{2, a, f}(x)}{(x - a)^2} &=  0
    \end{align*}
    Entonces 
    \begin{align*}
        \lim_{h \to 0} \frac{R_{2, a, f}(a - h)}{h^2} &= 0
    \end{align*}
    Consecuentemente
    \begin{align*}
        f''(a) + \lim_{h \to 0} \frac{R_{2, a, f}(a + h)}{h^2} + \lim_{h \to 0} \frac{R_{2, a, f}(a - h)}{h^2} &= f''(a) + 0 + 0 \\
        &= f''(a)
    \end{align*}
    Por lo tanto 
    \begin{align*}
        f''(a) &= \lim_{h \to 0} \frac{f(a + h) + f(a - h) -2f(a)}{h^2}
    \end{align*}
\end{proof}
\subsection{Ejercicio 2}
\noindent
Encuentra el valor de \(e\) hasta el primer decimal utilizando polinomios de Taylor.
\subsubsection*{Solución}
Consideremos \(f : \mathbb{R} \to \mathbb{R}\) definida como
\[
    f(x) = e^{x}
\]
Por definición sabemos que 
\[
    f(x) = P_{n, 0, f}(x) + R_{n, 0, f}(x)
\]
En particular, 
\begin{align*}
    e^{1} &= f(1) \\
    &= P_{n, 0, f}(1) + R_{n, 0, f}(1) \\
\end{align*}
Lo que buscamos es que 
\[
    | R_{n, 0, f}(x) | < 10^{-1}
\]
Buscamos \(10^{-1}\) ya que nos solicitan que sean decimales.
\newline
Recordando el polinomio de Taylor de grado \(n\) en \(0\) de \(e^{x}\) tenemos que
\[
    P_{n, 0, f}(x) = 1 + x + \frac{x^2}{2!} + \dotsc + \frac{x^n}{n!} = \sum_{k = 0}^{n} \frac{x^k}{k!}
\]
Es decir el término \(k-\)ésimo del polinomio de Taylor de grado \(n\) en \(0\) tenemos que
\[
    \frac{x^k}{k!}
\]
Recordemos que 
\begin{align*}
    \log{x} &= \int_{1}^{x} \frac{1}{t} dt \\
    e^{x} &= \log^{-1}{x} \\
\end{align*}
Así tenemos que
\[
    e^{1} = \log^{-1}{1}
\]
Lo que implica que
\[
    \log{e^{1}} = 1
\]
Por la definición de la función \(\log{x}\) tenemos que 
\[
    \int_{1}^{e} \frac{1}{t} dt = 1
\]
Notemos que
\[
    \int_{1}^{2} \frac{1}{t} dt  \leq 1
\]
Ya que si consideramos la partición \(\{1, 2\}\) del intervalo \([1, 2]\) y como \(\frac{1}{x}\)
es una función decreciente tenemos que sí nos tomamos la suma superior correspondiente a la 
partición tenemos que \(\sup \{\frac{1}{x} \ | \ x \in [1, 2]\} = \frac{1}{1} = 1 \) así la suma superior 
es \((2 - 1)(1) = (1)(1) = 1\) y por propiedades de la integral tenemos que \(\int_{1}^{3} \frac{1}{t} dt \leq 2\).
\newline
Notemos que 
\[
    2 \leq \int_{1}^{4} \frac{1}{t} dt  
\]
Ya que si consideramos la partición \(P = \{1, 2, 3, 4\}\) del intervalo \([1, 4]\) y como \(\frac{1}{x}\)
es una función decreciente tenemos que sí nos tomamos la suma inferior correspondiente a la 
partición tenemos que 
\begin{align*}
    L\left(\frac{1}{t}, P\right) &= \sum_{i=1}^{n}m_{i}(x_{i} - x_{i - 1})
\end{align*}
donde 
\begin{align*}
    m_{i} &= \inf \left\{ \frac{1}{t} \ | \ t \in [x_{i - 1}, x_{i} ]  \right\} = \frac{1}{x_{i-1}} \ \forall \ i \in \{1, \dotsc, 8\}
\end{align*}
Ya que \(\frac{1}{x}\) es decreciente. Consecuentemente
\begin{align*}
    L\left(\frac{1}{t}, P\right) &= (1)1 + (1)\frac{1}{2} + (1)\frac{1}{3} + (1)\frac{1}{4} \\
    &= 1 + \frac{1}{2} + \frac{1}{3} + \frac{1}{4} \\
    &= \frac{25}{12} \approx 2{.}08
\end{align*}
Así 
\[
    2 \leq \int_{1}^{8} \frac{1}{t} dt  
\]
Por lo tanto tenemos que 
\begin{align*}
    \int_{1}^{2} \frac{1}{t} dt \leq &1 \leq \int_{1}^{4} \frac{1}{t} dt \\
    \log{2} \leq &1 \leq \log{4} \\
    e^{\log{2}} \leq &e^{1} \leq e^{\log{4}} \\
    2 \leq &e^{1} \leq 4  \\
\end{align*}
Como \(e^{x}\) tiene derivadas en todos los órdenes para cada 
\(x \in \mathbb{R}\), es decir, \(e^{x}\) es de clase \(C^{\infty}\) en \(\mathbb{R}\).
en particular existe \(\left(e^{x}\right)^{(n + 1)}\) para cualquier \(n \in \mathbb{N}\).
Observamos que \(\left(e^{x}\right)^{(k)} = e^{x}\) para todo \(k \in \mathbb{N}\).
\newline 
Sea \(y = 1\) y \(a = 0\), tenemos que \(y \neq a\)
Por lo desarrollado anteriormente
\begin{align*}
    |f^{(n + 1)}(t)| &= |e^{t}| \\
                     &= e^{t} \\
                     &\leq e^{1} \\
                     &= e \\
                     &\leq 4 \\
\end{align*}
para toda \(t \in [0, 1]\).
Por un corolario del teorema de Taylor tenemos que 
\begin{align*}
    |R_{n, 0, e^{x}}(x)| &\leq \frac{4}{(n + 1)!}|1 - 0|^{n+1} \\
                         &\leq \frac{4}{(n + 1)!}|1|^{n+1} \\
                         &\leq \frac{4}{(n + 1)!}1^{n+1} \\
                         &\leq \frac{4}{(n + 1)!}
\end{align*}
Entonces lo que buscamos es
\begin{align*}
    \frac{4}{(n + 1)!} &< 10^{-1}
\end{align*}
La \(n\) que nos hace válida la desigualdad es \(n = 4\), ya que
\begin{align*}
    \frac{4}{(5 + 1)!} &= \frac{4}{(6)!} = \frac{1}{30} \leq \frac{1}{10}
\end{align*}
Por lo tanto
\begin{align*}
    e^{1} &= e \approx \sum_{k = 0}^{4} \frac{1^k}{k!} = 2{.}708\overline{3}
\end{align*}
\subsection{Ejercicio 3}
\noindent
Demuestra que el polinomio de Taylor de \(f(x) = \cos{x^2}\) de grado \(4n\) en 0 es
\[
    1 - \frac{x^4}{2!} + \frac{x^8}{4!} - \dotsc + (-1)^{n}\frac{x^{4n}}{(2n)!}
\]
\subsubsection*{Solución}
\begin{proof}
    Primero calcularemos el polinomio de Taylor de \(\cos{x}\).
    Por definición tenemos que el polinomio de Taylor de grado \(n\) de una función \(f\), en \(0\)
    es
    \begin{align*}
        P_{n, 0, f} = \sum_{k = 0}^{n} \frac{f^{(k)}(0)}{k!}x^{k}
    \end{align*}
    donde \(f^{(k)}\) es la \(k-\)ésima derivada de la función \(f\). En éste caso 
    \(f(x) = \cos{x}\), como \(\cos{x}\) tiene derivadas en todos los órdenes para cada 
    \(x \in \mathbb{R}\), es decir, \(\cos{x}\) es de clase \(C^{\infty}\) en \(\mathbb{R}\).
    \newline 
    Notemos lo siguiente:
    \begin{align*}
        f'(x) &= -\sin{x} \\
        f''(x) &= -\cos{x} \\
        f^{(3)}(x) &= \sin{x} \\
        f^{(4)}(x) &= \cos{x} \\
        f^{(5)}(x) &= -\sin{x} \\
        f^{(6)}(x) &= -\cos{x} \\
        f^{(7)}(x) &= \sin{x} \\
        f^{(8)}(x) &= \cos{x}
    \end{align*}
    En general tenemos que
    \begin{align*}
        f^{(4n)}(x) &= \cos{x} \\
        f^{(4n + 1)}(x) &= -\sin{x} \\
        f^{(4n + 2)}(x) &= -\cos{x} \\
        f^{(4n + 3)}(x) &= \sin{x} \\
    \end{align*}
    Observamos que
    \begin{align*}
        f^{(4n)}(0) &= 1 \\
        f^{(4n + 1)}(0) &= 0 \\
        f^{(4n + 2)}(0) &= -1 \\
        f^{(4n + 3)}(0) &= 0
    \end{align*}
    con \(n \in \mathbb{N} \cup \{0\}\).
    \newline
    Consecuentemente todos los términos impares nos van a quedar siempre \(0\). Para 
    facilitar el desarrollo del polinomio de Taylor eligiremos un grado par que son 
    los términos que no son \(0\). Por lo tanto,
    \begin{align*}
        P_{2n, 0, \cos{x}} &= \sum_{k=0}^{2n} \frac{f^{(k)}(0)}{k!}x^{k} \\
        &= f(0) + \frac{f'(0)}x + \frac{f''(0)}{2!}x^{2} + \frac{f^{(3)}(0)}{3!}x^{3} + \dotsc + \frac{f^{(2n - 1)}(0)}{(2n - 1)!}x^{2n - 1} +\frac{f^{(2n)}(0)}{(2n)!}x^{2n} \\
        &= 1 + 0 - \frac{x^2}{2!} + 0 + \frac{x^4}{4!} + 0 - \frac{x^6}{6!} + \dotsc + \frac{(-1)^{n}x^{2n}}{(2n!)} \\
        &= 1 - \frac{x^2}{2!} + \frac{x^4}{4!} - \frac{x^6}{6!} + \dotsc + \frac{(-1)^{n}x^{2n}}{(2n!)} \\
    \end{align*}
    Por definición tenemos que 
    \begin{align*}
        \cos_{x} &= P_{2n, 0, \cos{x}}(x) + R_{2n, 0, \cos{x}}(x) \\
    \end{align*}
    Sabemos que si \(f\) es una función tal que \(f'(a), \dotsc, f^{(n)}(a)\) existen. Si 
    \(P\) es un polinomio en \((x-a)\) de grado menor o igual a \(n\), igual a \(f\) 
    hasta el orden \(n\) en \(a\), entonces \(P = P_{n, a, f}\).
    Como \(\cos{x}\) tiene derivadas en todos los órdenes para cada 
    \(x \in \mathbb{R}\), es decir, \(\cos{x}\) es de clase \(C^{\infty}\) en \(\mathbb{R}\), 
    entonces \(\cos^{(0)}{x}, \cos^{(1)}{x}, \dotsc, \cos^{(2n)}{x}\) existen, tenemos que 
    \begin{align*}
        \lim_{x \to 0} \frac{\cos{x} - P_{2n, 0, \cos{x}}(x)}{x^{2n}} &= \lim_{x \to 0} \frac{R_{2n, 0, \cos{x}}(x)}{x^{2n}}\\
        &= 0
    \end{align*}
    Ya que \(P_{2n, 0, \cos{x}}\) es el polinomio de Taylor de \(\cos{x}\).
    \newline
    Lo anterior implica que
    \begin{align*}
        \lim_{x \to 0} \frac{R_{2n, 0, \cos{x}}(x^2)}{(x^2)^{2n}} &= 0
    \end{align*}
    Esto por definición es
    \begin{align*}
        \lim_{x \to 0} \frac{f(x^2) - P_{2n, 0, \cos{x}}(x^2)}{x^{4n}} &= 
        \lim_{x \to 0} \frac{\cos{x^2} - P_{2n, 0, \cos{x}}(x^2)}{x^{4n}} \\
    \end{align*}
    Sabemos que si \(f\) es una función tal que \(f'(a), \dotsc, f^{(n)}(a)\) existen. Si 
    \(P\) es un polinomio en \((x-a)\) de grado menor o igual a \(n\), igual a \(f\) 
    hasta el orden \(n\) en \(a\), entonces \(P = P_{n, a, f}\).
    Por lo tanto 
    \begin{align*}
        P_{4n, 0, \cos{x^2}}(x) = P_{2n, 0, \cos{x}}(x^2)
    \end{align*}
    Consecuentemente 
    \begin{align*}
        P_{4n, 0, \cos{x^2}}(x) = &= 1 - \frac{x^4}{2!} + \frac{x^8}{4!} - \frac{x^{12}}{6!} + \dotsc + \frac{(-1)^{n}x^{4n}}{(2n!)} \\
    \end{align*}
\end{proof}
\newpage
\subsection{Ejercicio 5}
\noindent
Aproximar los siguientes números con sumas cuyo error sea menor al indicado
\begin{itemize}
    \item \(\sin{1}\), error \(< 10^{-9}\).
    \item \(e^{2}\), error \(< 10^{-5}\)
\end{itemize}
\subsubsection*{Solución}
Veamos \(\sin{1}\).
\newline 
Consideremos \(f : \mathbb{R} \to \mathbb{R}\) definida como
\[
    f(x) = \sin{x}
\]
Por definición sabemos que 
\[
    f(x) = P_{n, 0, f}(x) + R_{n, 0, f}(x)
\]
En particular, 
\begin{align*}
    \sin{1} &= f(1) \\
    &= P_{n, 0, f}(1) + R_{n, 0, f}(1) \\
\end{align*}
Lo que buscamos es que 
\[
    | R_{n, 0, f}(x) | < 10^{-9}
\]
Recordando el polinomio de Taylor de grado \(2n + 1\) en \(0\) tenemos que
\[
    P_{2n + 1, 0, f}(x) = x - \frac{x^3}{3!} + \frac{x^5}{5!} - \frac{x^7}{7!} + \dotsc + \frac{(-1)^{n}}{(2n + 1)!}x^{2n + 1}
\]
Es decir el término \(k-\)ésimo del polinomio de Taylor de grado \(2n + 1\) en \(0\) tenemos que
\[
    \frac{(-1)^{k}}{(2k + 1)!}x^{2k + 1}
\]
Como \(\sin{x}\) tiene derivadas en todos los órdenes para cada 
\(x \in \mathbb{R}\), es decir, \(\sin{x}\) es de clase \(C^{\infty}\) en \(\mathbb{R}\).
en particular existe \(\sin^{(n+1)}{x}\), además sea \(y = 1\) y \(a = 0\), se tiene que 
\(y \geq a\), notemos lo siguiente:
\begin{align*}
    f'(x) &= \cos{x} \\
    f''(x) &= -\sin{x} \\
    f^{(3)}(x) &= -\cos{x} \\
    f^{(4)}(x) &= \sin{x} \\
    f^{(5)}(x) &=  \cos{x}  \\
    f^{(6)}(x) &= -\sin{x}  \\
    f^{(7)}(x) &= -\cos{x} \\
    f^{(8)}(x) &=  \sin{x} 
\end{align*}
En general tenemos que
\begin{align*}
    f^{(4n)}(x) &= \sin{x} \\
    f^{(4n + 1)}(x) &= \cos{x} \\
    f^{(4n + 2)}(x) &= -\sin{x} \\
    f^{(4n + 3)}(x) &= -\cos{x} \\
\end{align*}
Así tenemos que \(| f^{(n+1)}(t) | \leq 1\) para toda \(t \in [a, y]\) por un corolario del 
Teorema de Taylor tenemos que 
\begin{align*}
    | R_{n, 0, f}(t) | &\leq \frac{1}{(n + 1)!}|1|^{n} = \frac{1}{(n + 1)!}
\end{align*}
para cada \(t \in [a, y]\).
\newline 
Entonces lo que buscamos es que 
\[
    \frac{1}{(n + 1)!} < 10^{-9}
\]
La \(n\) que cumple con lo solicitado es \(n = 11\), ya que 
\begin{align*}
    \frac{1}{(11 + 1)!} = \frac{1}{(12)!} = 1{.}6059 \times 10^{-10} & < 10^{-9}
\end{align*}
Así 
\[
    \sin{1} \approx P_{11, 0, f}(1) = 1 - \frac{1}{3!} + \frac{1}{5!} - \frac{1}{7!} + \frac{1}{9!} - \frac{1}{11!}
    = 0{.}841470984808
\]
Veamos \(e^{2}\).
\newline 
Consideremos \(f : \mathbb{R} \to \mathbb{R}\) definida como
\[
    f(x) = e^{x}
\]
Por definición sabemos que 
\[
    f(x) = P_{n, 0, f}(x) + R_{n, 0, f}(x)
\]
En particular, 
\begin{align*}
    e^{2} &= f(2) \\
    &= P_{n, 0, f}(2) + R_{n, 0, f}(2) \\
\end{align*}
Lo que buscamos es que 
\[
    | R_{n, 0, f}(x) | < 10^{-5}
\]
Recordando el polinomio de Taylor de grado \(n\) en \(0\) de \(e^{x}\) tenemos que
\[
    P_{n, 0, f}(x) = 1 + x + \frac{x^2}{2!} + \dotsc + \frac{x^n}{n!} = \sum_{k = 0}^{n} \frac{x^k}{k!}
\]
Es decir el término \(k-\)ésimo del polinomio de Taylor de grado \(n\) en \(0\) tenemos que
\[
    \frac{x^k}{k!}
\]
Recordemos que 
\begin{align*}
    \log{x} &= \int_{1}^{x} \frac{1}{t} dt \\
    e^{x} &= \log^{-1}{x} \\
\end{align*}
Así tenemos que
\[
    e^{2} = \log^{-1}{2}
\]
Lo que implica que
\[
    \log{e^{2}} = 2
\]
Por la definición de la función \(\log{x}\) tenemos que 
\[
    \int_{1}^{e^{2}} \frac{1}{t} dt = 2
\]
Notemos que
\[
    \int_{1}^{3} \frac{1}{t} dt  \leq 2
\]
Ya que si consideramos la partición \(\{1, 3\}\) del intervalo \([1, 3]\) y como \(\frac{1}{x}\)
es una función decreciente tenemos que sí nos tomamos la suma superior correspondiente a la 
partición tenemos que \(\sup \{\frac{1}{x} \ | \ x \in [1, 3]\} = \frac{1}{1} = 1 \) así la suma superior 
es \((3 - 1)(1) = (2)(1) = 2\) y por propiedades de la integral tenemos que \(\int_{1}^{3} \frac{1}{t} dt  \leq 2\).
Notemos que 
\[
    2 \leq \int_{1}^{8} \frac{1}{t} dt  
\]
Ya que si consideramos la partición \(P = \{1, 2, 3, 4, 5, 6, 7, 8\}\) del intervalo \([1, 8]\) y como \(\frac{1}{x}\)
es una función decreciente tenemos que sí nos tomamos la suma inferior correspondiente a la 
partición tenemos que 
\begin{align*}
    L\left(\frac{1}{t}, P\right) &= \sum_{i=1}^{n}m_{i}(x_{i} - x_{i - 1})
\end{align*}
donde 
\begin{align*}
    m_{i} &= \inf \left\{ \frac{1}{t} \ | \ t \in [x_{i - 1}, x_{i} ]  \right\} = \frac{1}{x_{i-1}} \ \forall \ i \in \{1, \dotsc, 8\}
\end{align*}
Ya que \(\frac{1}{x}\) es decreciente. Consecuentemente
\begin{align*}
    L\left(\frac{1}{t}, P\right) &= (1)1 + (1)\frac{1}{2} + (1)\frac{1}{3} + (1)\frac{1}{4} + (1)\frac{1}{5} + (1)\frac{1}{6} + (1)\frac{1}{7} \\
    &= 1 + \frac{1}{2} + \frac{1}{3} + \frac{1}{4} + \frac{1}{5} + \frac{1}{6} + \frac{1}{7} \\
    &= \frac{363}{140} \approx 2{.}59
\end{align*}
Así 
\[
    2 \leq \int_{1}^{8} \frac{1}{t} dt  
\]
Por lo tanto tenemos que 
\begin{align*}
    \int_{1}^{3} \frac{1}{t} dt \leq &2 \leq \int_{1}^{8} \frac{1}{t} dt \\
    \log{3} \leq &2 \leq \log{8} \\
    e^{\log{3}} \leq &e^{2} \leq e^{\log{8}} \\
    3 \leq &e^{2} \leq 8  \\
\end{align*}
Como \(e^{x}\) tiene derivadas en todos los órdenes para cada 
\(x \in \mathbb{R}\), es decir, \(e^{x}\) es de clase \(C^{\infty}\) en \(\mathbb{R}\).
en particular existe \(\left(e^{x}\right)^{(n + 1)}\) para cualquier \(n \in \mathbb{N}\).
Observamos que \(\left(e^{x}\right)^{(k)} = e^{x}\) para todo \(k \in \mathbb{N}\).
\newline 
Sea \(y = 2\) y \(a = 0\), tenemos que \(y \neq a\)
Por lo desarrollado anteriormente
\begin{align*}
    |f^{(n + 1)}(t)| &= |e^{t}| \\
                     &= e^{t} \\
                     &\leq e^{2} \\
                     &\leq 8 \\
\end{align*}
para toda \(t \in [0, 2]\).
\newline 
Por un corolario del teorema de Taylor tenemos que 
\begin{align*}
    |R_{n, 0, e^{2}}(x)| &\leq \frac{8}{(n + 1)!}|2 - 0|^{n+1} \\
                         &\leq \frac{8}{(n + 1)!}|2|^{n+1} \\
                         &\leq \frac{8}{(n + 1)!}2^{n+1} \\
                         &\leq \frac{2^{3}}{(n + 1)!}2^{n+1} \\
                         &\leq \frac{2^{3}2^{n+1}}{(n + 1)!} \\
                         &\leq \frac{2^{n+4}}{(n + 1)!} \\
\end{align*}
Entonces lo que buscamos es
\begin{align*}
    \frac{2^{n+4}}{(n + 1)!} &< 10^{-5}
\end{align*}
La \(n\) que nos hace válida la desigualdad es \(n = 13\), ya que
\begin{align*}
    \frac{2^{13+4}}{(13 + 1)!} &= \frac{2^{17}}{(14)!} \\
                               &\approx 1{.}504 \times 10^{-6} \\
                               &< 10^{-5}
\end{align*}
Por lo tanto
\begin{align*}
    e^{2} &\approx \sum_{k = 0}^{13} \frac{2^k}{k!} = 7{.}38905
\end{align*}
\end{document}