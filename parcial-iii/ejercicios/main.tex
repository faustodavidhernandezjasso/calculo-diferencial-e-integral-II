\documentclass[a4paper]{article} 
\input{head}
\newcommand{\pow}[2]{#1^{#2}}
\newcommand{\supra}[1]{\textsuperscript{#1}}
\begin{document}

%-------------------------------
%	TITLE SECTION
%-------------------------------

\fancyhead[C]{}
\hrule \medskip % Upper rule
\begin{minipage}{0.35\textwidth} 
\raggedright
\footnotesize
Fausto David Hernández Jasso \hfill\\   
317000928 \hfill\\
fausto.david.hernandez.jasso@ciencias.unam.mx
\end{minipage}
\begin{minipage}{0.4\textwidth} 
\centering 
\large 
Cálculo Diferencial e Integral II\\ 
\normalsize 
Métodos de integración\\ 
\end{minipage}
\begin{minipage}{0.24\textwidth} 
\raggedleft
\today\hfill\\
\end{minipage}
\medskip\hrule 
\bigskip
\section{Ejercicios}
\subsection{Ejercicio 1}
\noindent
Demuestre que
\begin{align*}
    \int_{1}^{\cosh{x}} \sqrt{t^2 - 1} \ dt = \frac{\cosh{x}\sinh{x}}{2} - \frac{x}{2}
\end{align*}
\begin{proof}
    Aplicando la siguiente sustitución trigonométrica a la integral indefinida \(\int \sqrt{t^2 - 1} \ dt\).
    \begin{align*}
        t &= \sec{\theta} \\
        dt &= \sec{\theta} \tan{\theta} \\
    \end{align*}
    Consecuentemente 
    \begin{align*}
        \int \sqrt{t^2 - 1} \ dt &= \int \sqrt{\left(\sec{\theta}\right)^2 - 1}\sec{\theta} \tan{\theta} \ d\theta \\
                                 &= \int \sqrt{\left(\tan{\theta}\right)^2}\sec{\theta} \tan{\theta} \ d\theta \\
                                 &= \int \tan{\theta}\sec{\theta} \tan{\theta} \ d\theta \\
                                 &= \int \tan^2{\theta}\sec{\theta} \ d\theta                                 
    \end{align*}
    Como \(\tan^2{\theta} = \sec^{2}{\theta} - 1\) entonces 
    \begin{align*}
        \int \tan^2{\theta}\sec{\theta} \ d\theta &= \int \left(\sec^{2}{\theta} - 1\right)\sec{\theta} \ d\theta \\
                                                  &= \int \sec^{3}{\theta} - \sec{\theta} \ d\theta \\
                                                  &= \int \sec^{3}{\theta} \ d\theta - \int \sec{\theta} \ d\theta
    \end{align*}
    Como
    \begin{align*}
        \int \sec^{3}{\theta} \ d\theta &= \frac{1}{2}\left(\sec{\theta}\tan{\theta} + \log{\left(\sec{\theta} + \tan{\theta}\right)}\right) + C \\
        \int \sec{\theta} \ d\theta &= \log{\left(\sec{\theta} + \tan{\theta}\right)} + C
    \end{align*}
    Entonces 
    \begin{align*}
        \int \sec^{3}{\theta} \ d\theta - \int \sec{\theta} \ d\theta &= 
        \frac{1}{2}\left(\sec{\theta}\tan{\theta} + \log{\left(\sec{\theta} + \tan{\theta}\right)}\right)
        - 
        \log{\left(\sec{\theta} + \tan{\theta}\right)} + C \\
        &= \frac{1}{2}\sec{\theta}\tan{\theta} - \frac{1}{2}\log{\left(\sec{\theta} + \tan{\theta}\right)} + C
    \end{align*}
    Tenemos que
    \begin{align*}
        \sec{\theta} &= t \\
        \tan{\theta} &= \sqrt{t^2 - 1}
    \end{align*}
    Entonces 
    \begin{align*}
        \frac{1}{2}\sec{\theta}\tan{\theta} - \frac{1}{2}\log{\left(\sec{\theta} + \tan{\theta}\right)} + C &= 
        \frac{1}{2}t\sqrt{t^2 - 1} - \frac{1}{2}\log{\left(t + \sqrt{t^2 - 1}\right)} + C
    \end{align*}
    Así tenemos que 
    \begin{align*}
        \int_{1}^{\cosh{x}} \sqrt{t^2 - 1} \ dt &= \left(\frac{1}{2}t\sqrt{t^2 - 1} - \frac{1}{2}\log{\left(t + \sqrt{t^2 - 1}\right)}\right) \Big|_1^{\cosh{x}} \\
        &= \left(\frac{1}{2}(\cosh{x})\sqrt{(\cosh{x})^2 - 1} - \frac{1}{2}\log{\left((\cosh{x}) + \sqrt{(\cosh{x})^2 - 1}\right)}\right) - \\
           &\left(\frac{1}{2}1\sqrt{1^2 - 1} - \frac{1}{2}\log{\left(1 + \sqrt{1^2 - 1}\right)}\right) \\
        &= \frac{1}{2}(\cosh{x})\sqrt{(\cosh{x})^2 - 1} - \frac{1}{2}\log{\left(\cosh{x} + \sqrt{(\cosh{x})^2 - 1}\right)}
    \end{align*}
    Sabemos que \((\cosh{x})^2 - 1 = \sinh^{2}{x}\) entonces
    \begin{align*}
        \frac{1}{2}(\cosh{x})\sqrt{(\cosh{x})^2 - 1} - \frac{1}{2}\log{\left(\cosh{x} + \sqrt{(\cosh{x})^2 - 1}\right)} &= 
        \frac{1}{2}(\cosh{x})\sqrt{(\sinh{x})^2} - \frac{1}{2}\log{\left(\cosh{x} + \sqrt{(\sinh{x})^2}\right)} \\
        &= \frac{1}{2}\cosh{x}\sinh{x} - \frac{1}{2}\log{\left(\cosh{x} + \sinh{x}\right)}
    \end{align*}
    Recordando la definición de \(\sinh{x}\) y \(\cosh{x}\)
    \begin{align*}
        \sinh{x} &= \frac{e^{x} - e^{-x}}{2} \\
        \cosh{x} &= \frac{e^{x} + e^{-x}}{2}
    \end{align*}
    Sustituyendo
    \begin{align*}
        \frac{1}{2}\cosh{x}\sinh{x} - \frac{1}{2}\log{\left(\cosh{x} + \sinh{x}\right)} &=
        \frac{1}{2}\cosh{x}\sinh{x} - \frac{1}{2}\log{\left(\frac{e^{x} + e^{-x}}{2} + \frac{e^{x} - e^{-x}}{2}\right)} \\
        &= \frac{1}{2}\cosh{x}\sinh{x} - \frac{1}{2}\log{\left(\frac{2e^{x}}{2}\right)} \\
        &= \frac{1}{2}\cosh{x}\sinh{x} - \frac{1}{2}\log{\left(e^{x}\right)} \\
        &= \frac{1}{2}\cosh{x}\sinh{x} - \frac{1}{2}x \\
        &= \frac{\cosh{x}\sinh{x}}{2} - \frac{x}{2} \\
    \end{align*}
    Por lo tanto 
    \begin{align*}
        \int_{1}^{\cosh{x}} \sqrt{t^2 - 1} \ dt &= \frac{\cosh{x}\sinh{x}}{2} - \frac{x}{2}
    \end{align*}
\end{proof}
\newpage
\subsection{Ejercicio 2}
\noindent
Demuestre que 
\[
    \int_{a}^{b} f(x) dx = \int_{a}^{b} f\left(a + b - x\right) dx
\]
\begin{proof}
    Definimos a \(g(x) = a + b - x\), notemos que \(g'(x) = - 1\) que es continua. 
    Notemos que \(\left(f \circ g\right)(x) = f(g(x)) = f(a + b - x)\). 
    \begin{align*}
        \int_{a}^{b} f(a + b - x) dx &= \int_{a}^{b} f(a + b - x) (-1)(-1) dx \\
                                     &= - \int_{a}^{b} f(a + b - x) (-1) dx \\
                                     &= - \int_{a}^{b} f(a + b - x) g'(x) dx \\
                                     &= - \int_{a}^{b} f(g(x)) g'(x) dx
    \end{align*}
    Por el \textbf{Teorema de Sustitución} tenemos que:
    \begin{align*}
        - \int_{a}^{b} f(g(x)) g'(x) dx &= - \int_{g(a)}^{g(b)} f(x) dx \\
                                      &= - \int_{b}^{a} f(x) dx \\
                                      &= \int_{a}^{b} f(x) dx
    \end{align*}
\end{proof}
\newpage
\subsection{Ejercicio 3}
\noindent
Suponga que \(f''\) es continua y que
\[
    \int_{0}^{\pi} \left[f(x) + f''(x)\right]\sin{x} dx = 2
\]
Dado que \(f\left(\pi\right) = 1\), calcule \(f(0)\)
\begin{proof}
    Como \(f\) tiene al menos segunda derivada, entonces sabemos que
    \begin{itemize}
        \item \(f\) es continua, ya que existe \(f'\).
        \item \(f'\) es continua, ya que existe \(f''\).
    \end{itemize}
    Además por hipótesis tenemos que \(f''\) es continua. Sabemos que \(\sin{x}\) y \(\cos{x}\) son 
    funciones continuas, consecuentemente podemos aplicar el teorema de \textbf{integración por partes}.
    \begin{align*}
        \int_{0}^{\pi} \left[f(x) + f''(x)\right]\sin{x} dx &= 2 \\
        \int_{0}^{\pi} f(x)\sin{x} + f''(x)\sin{x} dx &= 2 \\
        \int_{0}^{\pi} f(x)\sin{x} dx + \int_{0}^{\pi} f''(x)\sin{x} dx &= 2 \\
        \int_{0}^{\pi} f''(x)\sin{x} dx &= 2 - \int_{0}^{\pi} f(x)\sin{x} dx \\
    \end{align*}
    Aplicaremos integración por partes a la integral \(\int_{0}^{\pi} f(x)\sin{x} dx\)
    donde:
    \begin{itemize}
        \item \(u = f(x)\)
        \item \(du = f'(x) \ dx\)
        \item \(v = -\cos{x}\)
        \item \(dv = \sin{x} \ dx\)
    \end{itemize}
    Así 
    \begin{align*}
        \int_{0}^{\pi} f(x)\sin{x} dx &= -f(x)\cos{x} \Big|_0^\pi - \int_{0}^{\pi} f'(x)-\cos{x} dx \\
                                      &= -f(x)\cos{x} \Big|_0^\pi + \int_{0}^{\pi} f'(x)\cos{x} dx
    \end{align*}
    Aplicaremos integración por partes a la integral \(\int_{0}^{\pi} f'(x)\cos{x} dx\)
    donde:
    \begin{itemize}
        \item \(u_1 = f'(x)\)
        \item \(du_1 = f''(x) \ dx\)
        \item \(v_1 = \sin{x}\)
        \item \(dv_1 = \cos{x} \ dx\)
    \end{itemize}
    Así
    \begin{align*}
        \int_{0}^{\pi} f'(x)\cos{x} dx &= f'(x)\sin{x} \Big|_0^\pi - \int_{0}^{\pi} f''(x)\sin{x} dx
    \end{align*}
    Sustituyendo tenemos que
    \begin{align*}
        \int_{0}^{\pi} f''(x)\sin{x} dx &= 2 - \int_{0}^{\pi} f(x)\sin{x} dx \\
                                        &= 2 - \left[ -f(x)\cos{x} \Big|_0^\pi + \int_{0}^{\pi} f'(x)\cos{x} dx \right] \\
                                        &= 2 + f(x)\cos{x} \Big|_0^\pi - \int_{0}^{\pi} f'(x)\cos{x} dx \\
                                        &= 2 + f(x)\cos{x} \Big|_0^\pi - \left[ f'(x)\sin{x}\Big|_0^\pi - \int_{0}^{\pi} f''(x)\sin{x} dx \right] \\
                                        &= 2 + f(x)\cos{x} \Big|_0^\pi - f'(x)\sin{x}\Big|_0^\pi + \int_{0}^{\pi} f''(x)\sin{x} dx \\
    \end{align*}
    Obtenemos que 
    \begin{align*}
        \int_{0}^{\pi} f''(x)\sin{x} dx &= 2 + f(x)\cos{x} \Big|_0^\pi - f'(x)\sin{x}\Big|_0^\pi + \int_{0}^{\pi} f''(x)\sin{x} dx \\
        0 &= 2 + f(x)\cos{x} \Big|_0^\pi - f'(x)\sin{x}\Big|_0^\pi \\
        0 &= 2 + f(\pi)\cos{\pi} - f(0)\cos{0} - f'(\pi)\sin{\pi} + f'(0)\sin{0} \\
        0 &= 2 + f(\pi)\cos{\pi} - f(0)\cos{0} - f'(\pi)(0) + f'(0)(0) \\
        0 &= 2 + f(\pi)\cos{\pi} - f(0)\cos{0} - 0 + 0 \\
        0 &= 2 + f(\pi)\cos{\pi} - f(0)\cos{0} \\
        0 &= 2 + f(\pi)\cos{\pi} - f(0)\cos{0} \\
        f(0)\cos{0} &= 2 + f(\pi)\cos{\pi} \\
        f(0)(1) &= 2 + f(\pi)(-1) \\
        f(0) &= 2 - f(\pi) \\
        f(0) &= 2 - 1 \\
        f(0) &= 1 \\
    \end{align*}
\end{proof}
\end{document}