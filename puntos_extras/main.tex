\documentclass[a4paper]{article} 
\input{head}
\newcommand{\pow}[2]{#1^{#2}}
\newcommand{\supra}[1]{\textsuperscript{#1}}
\begin{document}

%-------------------------------
%	TITLE SECTION
%-------------------------------

\fancyhead[C]{}
\hrule \medskip % Upper rule
\begin{minipage}{0.35\textwidth} 
\raggedright
\footnotesize
Fausto David Hernández Jasso \hfill\\   
317000928 \hfill\\
fausto.david.hernandez.jasso@ciencias.unam.mx
\end{minipage}
\begin{minipage}{0.4\textwidth} 
\centering 
\large 
Cálculo Diferencial e Integral II\\ 
\normalsize 
Aproximaciones con Polinomios\\ 
\end{minipage}
\begin{minipage}{0.24\textwidth} 
\raggedleft
\today\hfill\\
\end{minipage}
\medskip\hrule 
\bigskip
\section{Ejercicios}
\subsection{Ejercicio 1}
\noindent
Demuestra que el polinomio de Taylor de \(f(x) = \sin{x^2}\) de grado \(4n + 2\) en \(0\)
es
\[
    x^2 - \frac{x^6}{3!} + \frac{x^{10}}{5!} - \dotsc + (-1)^n \frac{x^{4n + 2}}{(2n + 1)!}
\]
\subsubsection*{Solución}
La definición del polinomio de Taylor en \(0\) de grado \(2n + 1\) de la función \(\sin{x}\)
\begin{align*}
    P_{2n + 1, 0, \sin{x}}(x) &= x - \frac{x^3}{3!} + \frac{x^5}{5!} - \frac{x^7}{7!} + \dotsc + \frac{(-1)^n}{(2n + 1)!}x^{2n+1}
\end{align*}
Por definición tenemos que 
\begin{align*}
    \sin{x} &= P_{2n + 1, 0, \sin{x}}(x) + R_{2n + 1, 0, \sin{x}}(x)
\end{align*}
Sabemos que si \(f\) es una función tal que \(f'(a), \dotsc, f^{(n)}(a)\) existen. Si 
\(P\) es un polinomio en \((x-a)\) de grado menor o igual a \(n\), igual a \(f\) 
hasta el orden \(n\) en \(a\), entonces \(P = P_{n, a, f}\).
Como \(\sin{x}\) tiene derivadas en todos los órdenes para cada 
\(x \in \mathbb{R}\), es decir, \(\sin{x}\) es de clase \(C^{\infty}\) en \(\mathbb{R}\), 
entonces \(\sin^{(0)}{x}, \sin^{(1)}{x}, \dotsc, \sin^{(2n + 1)}{x}\) existen, tenemos que 
\begin{align*}
    \lim_{x \to 0} \frac{\sin{x} - P_{2n + 1, 0, \sin{x}}(x)}{x^{2n + 1}} &= \lim_{x \to 0} \frac{R_{2n + 1, 0, \sin{x}}(x)}{x^{2n + 1}}\\
    &= 0
\end{align*}
Ya que \(P_{2n + 1, 0, \sin{x}}\) es el polinomio de Taylor de \(\sin{x}\).
\newline
Lo anterior implica que
\begin{align*}
    \lim_{x \to 0} \frac{R_{2n + 1, 0, \sin{x}}(x^2)}{(x^2)^{2n + 1}} &= 0
\end{align*}
Esto por definición es
\begin{align*}
    \lim_{x \to 0} \frac{\sin{x^2} - P_{2n + 1, 0, \sin{x}}(x^2)}{x^{4n + 2}} &= 0
\end{align*}
Sabemos que si \(f\) es una función tal que \(f'(a), \dotsc, f^{(n)}(a)\) existen. Si 
\(P\) es un polinomio en \((x-a)\) de grado menor o igual a \(n\), igual a \(f\) 
hasta el orden \(n\) en \(a\), entonces \(P = P_{n, a, f}\).
Por lo tanto 
\begin{align*}
    P_{4n + 2, 0, \sin{x^2}}(x) = P_{2n, 0, \sin{x}}(x^2)
\end{align*}
Consecuentemente 
\begin{align*}
    P_{4n + 2, 0, \sin{x^2}}(x) &= x^2 - \frac{x^6}{3!} + \frac{x^{10}}{5!} - \frac{x^{14}}{7!} + \dotsc + \frac{(-1)^n}{(2n + 1)!}x^{4n+2} \\
\end{align*}
\newpage
\subsection{Ejercicio 2}
\noindent
Se define para todo número real \(\alpha\) y para todo entero no negativo \(n\), el coeficiente 
binomial
\[
    \binom{\alpha}{n} = \frac{\alpha \cdot \left(\alpha - 1\right) \dotsc \left(\alpha - n + 1\right)}{n!}
\]
Demuestra que el polinomio de Taylor de la función \(f(x) = \left(1 + x\right)^{\alpha}\) en \(0\)
es
\[
    P_{n, 0}(x) = \sum_{k = 0}^{n} \binom{\alpha}{k} x^{k}
\]
\subsubsection*{Solución}
\begin{proof}
    Sea \(f(x) = \left(1 + x\right)^{\alpha}\) tenemos que 
    \begin{align*}
        f^{(1)}(x) &= \alpha\left(1 + x\right)^{\alpha - 1} \\
        f^{(2)}(x) &= \alpha\left(\alpha - 1\right)\left(1 + x\right)^{\alpha - 2} \\
        f^{(3)}(x) &= \alpha\left(\alpha - 1\right)\left(\alpha - 2\right)\left(1 + x\right)^{\alpha - 3} \\
        &\vdots \\
        f^{(n)}(x) &= \alpha\left(\alpha - 1\right)\left(\alpha - 2\right) \cdots \left(\alpha - n + 1\right)\left(1 + x\right)^{\alpha - n}
    \end{align*}
    Evaluando todas las derivadas en \(0\) tenemos que 
    \begin{align*}
        f^{(1)}(0) &= \alpha\left(1\right)^{\alpha - 1} = \alpha \\
        f^{(2)}(0) &= \alpha\left(\alpha - 1\right)\left(1\right)^{\alpha - 2} = \alpha\left(\alpha - 1\right) \\
        f^{(3)}(0) &= \alpha\left(\alpha - 1\right)\left(\alpha - 2\right)\left(1\right)^{\alpha - 3} = \alpha\left(\alpha - 1\right)\left(\alpha - 2\right) \\
        &\vdots \\
        f^{(n)}(0) &= \alpha\left(\alpha - 1\right)\left(\alpha - 2\right) \cdots \left(\alpha - n + 1\right)\left(1\right)^{\alpha - n} = \alpha\left(\alpha - 1\right)\left(\alpha - 2\right) \cdots \left(\alpha - n + 1\right)
    \end{align*}
    Por definición del polinomio de Taylor de \(f\) en \(0\)
    \begin{align*}
        P_{n, 0, f}(x) &= \sum_{k = 0}^{n} \frac{f^{(k)}(0)}{k!}x^k \\
        &= f(0) + \frac{f^{(1)}(0)}{1!}x + \frac{f^{(2)}(0)}{2!}x^2 + \frac{f^{(3)}(0)}{3!}x^3 + \dotsc + \frac{f^{(n)}(0)}{n!}x^n \\
        &= 1 + \frac{\alpha}{1!}x + \frac{\alpha\left(\alpha - 1\right)}{2!}x^2 + \frac{\alpha\left(\alpha - 1\right)\left(\alpha - 2\right)}{3!}x^3 + \dotsc + \frac{\alpha\left(\alpha - 1\right)\left(\alpha - 2\right) \cdots \left(\alpha - n + 1\right)}{n!}x^n \\
        &= 1\binom{\alpha}{0} + \binom{\alpha}{1}x + \binom{\alpha}{2}x^2 + \binom{\alpha}{3}x^3 + \dotsc + \binom{\alpha}{n}x^n \\
        &= \sum_{k = 0}^{n} \binom{\alpha}{k}x^{k}
    \end{align*}
\end{proof}
\newpage
\subsection{Ejercicio 3}
\noindent
Supongáse que \(a_{i}\) y \(b_{i}\) son los coeficientes de los polinomios de Taylor en \(a\) de \(f\)
y de \(g\), respectivamente. Hallar los coeficientes \(c_{i}\) de los polinomios de Taylos en \(a\)
de las siguientes funciones, en términos de \(a_{i}\) y \(b_{i}\).
\begin{itemize}
    \item \(f + g\)
    \item \(fg\)
    \item \(f'\)
    \item \(h(x) = \displaystyle \int_{0}^{x} f(t) dt\)
\end{itemize}
\subsubsection*{Solución}
Recordemos que los coeficientes \(a_{i}\) y \(b_{i}\) de los polinomios de Taylor en \(a\) de \(f\) y \(g\)
respectivamente están definidos como sigue:
\begin{align*}
    a_{i} &= \frac{f^{(i)}(a)}{i!} \\
    b_{i} &= \frac{g^{(i)}(a)}{i!}
\end{align*}
Veamos \(f + g\)
\newline
En éste caso por definición de la suma de polinomios, tenemos que 
\begin{align*}
    c_{i} &= a_{i} + b_{i}
\end{align*}
Veamos \(fg\)
\newline 
Notemos lo siguiente
\begin{align*}
    \left(fg\right)' &= fg' + gf' \\
    \left(fg\right)'' &= fg'' + f'g' + + f'g' + f''g = fg'' + 2f'g' + f''g \\
    \left(fg\right)''' &= fg''' + f'g'' + 2f''g' + 2f'g'' + f''g' + f'''g = f'''g + 3f''g' + 3f'g'' + fg''' \\
    &\vdots \\
    \left(fg\right)^{(n)} &= \binom{n}{0} f^{(n)}g^{(0)} + \binom{n}{1} f^{(n-1)}g^{(1)} + \binom{n}{2} f^{(n - 2)}g^{(2)} + \dotsc + \binom{n}{n - 2} f^{(2)}g^{(n - 2)} \binom{n}{n - 1} f^{(1)}g^{(n - 1)} + \binom{n}{n} f^{(0)}g^{(n)}
\end{align*}
Por definición del polinomio de Taylor de \(fg\) en \(a\), tenemos que 
\begin{align*}
    P_{n, a, fg} &= \sum_{k = 0}^{n} \frac{\left(fg\right)^{(k)}(a)}{k!}(x - a)^k
\end{align*}
Así tenemos que el \(k-\)ésimo término del polinomio de Taylor de \(fg\) en \(a\)
es
\begin{align*}
    c_{k} = \frac{\left(fg\right)^{(k)}(a)}{k!} &= \frac{\displaystyle \sum_{i = 0}^{k} \binom{k}{i}f^{(k - i)}(a)g^{(i)}(a)}{k!}
    = \sum_{i = 0}^{k} \binom{k}{i}a_{k - i}b_{i}
\end{align*}
Veamos \(f'\)
\newline
Por definición del polinomio de Taylor de \(f'\) en \(a\), tenemos que 
\begin{align*}
    P_{n, a, f'} &= \sum_{k = 0}^{n} \frac{\left(f'\right)^{(k)}(a)}{k!}(x - a)^k \\ 
                 &= \sum_{k = 0}^{n} \frac{f^{(k +1)}(a)}{k!}(x - a)^k \\
                 &= \frac{f^{(1)}(a)}{0!}(x - a)^0 + \frac{f^{(2)}(a)}{1!}(x - a)^1 + \frac{f^{(3)}(a)}{2!}(x - a)^2 + \dotsc + \frac{f^{(n)}(a)}{(n - 1)!}(x - a)^{n - 1} + \frac{f^{(n + 1)}(a)}{n!}(x - a)^n \\
                 &= f^{(1)}(a) + f^{(2)}(a)(x - a)^1 + \frac{f^{(3)}(a)}{2}(x - a)^2 + \dotsc + \frac{f^{(n)}(a)}{(n - 1)!}(x - a)^{n - 1} + \frac{f^{(n + 1)}(a)}{n!}(x - a)^n
\end{align*}
Así tenemos que 
\[
    c_{k} = \frac{f^{(k + 1)}(a)}{k!} = (k + 1)a_{k + 1}
\]
Veamos \(h(x) = \displaystyle \int_{0}^{x} f(t) dt\)
\newline 
Notemos que 
\begin{align*}
    h^{(1)}(x) &= f(x) \\
    h^{(2)}(x) &= f^{(1)}(x) \\
    h^{(3)}(x) &= f^{(2)}(x) \\
    h^{(4)}(x) &= f^{(3)}(x) \\
    &\vdots \\
    h^{(n)}(x) &= f^{(n - 1)}(x)
\end{align*} 
Por definición del polinomio de Taylor de \(h\) en \(a\), tenemos que 
\begin{align*}
    P_{n, a, h} &= \sum_{k = 0}^{n} \frac{\left(h\right)^{(k)}(a)}{k!}(x - a)^k \\ 
                 &= \int_{0}^{a} f(t) dt + \sum_{k = 1}^{n} \frac{f^{(k - 1)}(a)}{k!}(x - a)^k
\end{align*}
Así 
\begin{align*}
    c_{0} &= \int_{0}^{a} f(t) dt \\
    c_{i} &= a_{i - 1}
\end{align*}
Para toda \(1 \leq i \leq n\)
\end{document}