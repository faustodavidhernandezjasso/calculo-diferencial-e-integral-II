\documentclass[a4paper]{article} 
\addtolength{\hoffset}{-2.25cm}
\addtolength{\textwidth}{4.5cm}
\addtolength{\voffset}{-3.25cm}
\addtolength{\textheight}{5cm}
\setlength{\parskip}{0pt}
\setlength{\parindent}{0in}

%----------------------------------------------------------------------------------------
%	PACKAGES AND OTHER DOCUMENT CONFIGURATIONS
%----------------------------------------------------------------------------------------
\usepackage{mathtools} % Package for using math tools
\usepackage[inline]{enumitem} % Enumerate environment
\usepackage{blindtext} % Package to generate dummy text
\usepackage{charter} % Use the Charter font
\usepackage[utf8]{inputenc} % Use UTF-8 encoding
\usepackage{microtype} % Slightly tweak font spacing for aesthetics
\usepackage[spanish]{babel} % Language hyphenation and typographical rules
\usepackage{amsthm, amsmath, amssymb} % Mathematical typesetting
\usepackage{float} % Improved interface for floating objects
\usepackage[final, colorlinks = true, 
            linkcolor = black, 
            citecolor = black]{hyperref} % For hyperlinks in the PDF
\usepackage{graphicx, multicol} % Enhanced support for graphics
\usepackage{xcolor} % Driver-independent color extensions
\usepackage{marvosym, wasysym} % More symbols
\usepackage{rotating} % Rotation tools
\usepackage{censor} % Facilities for controlling restricted text
\usepackage{listings, style/lstlisting} % Environment for non-formatted code, !uses style file!
\usepackage{pseudocode} % Environment for specifying algorithms in a natural way
\usepackage{style/avm} % Environment for f-structures, !uses style file!
\usepackage{booktabs} % Enhances quality of tables
\usepackage{tikz-qtree} % Easy tree drawing tool
\tikzset{every tree node/.style={align=center,anchor=north},
         level distance=2cm} % Configuration for q-trees
\usepackage{style/btree} % Configuration for b-trees and b+-trees, !uses style file!
\usepackage[backend=biber,style=numeric,
            sorting=nyt]{biblatex} % Complete reimplementation of bibliographic facilities
\addbibresource{ecl.bib}
\usepackage{csquotes} % Context sensitive quotation facilities
\usepackage[yyyymmdd]{datetime} % Uses YEAR-MONTH-DAY format for dates
\renewcommand{\dateseparator}{-} % Sets dateseparator to '-'
\usepackage{fancyhdr} % Headers and footers
\usepackage{physics}
\pagestyle{fancy} % All pages have headers and footers
\fancyhead{}\renewcommand{\headrulewidth}{0pt} % Blank out the default header
\fancyfoot[L]{} % Custom footer text
\fancyfoot[C]{} % Custom footer text
\fancyfoot[R]{\thepage} % Custom footer text
\newcommand{\note}[1]{\marginpar{\scriptsize \textcolor{red}{#1}}} % Enables comments in red on margin


%----------------------------------------------------------------------------------------

\newcommand{\pow}[2]{#1^{#2}}
\newcommand{\supra}[1]{\textsuperscript{#1}}
\begin{document}

%-------------------------------
%	TITLE SECTION
%-------------------------------

\fancyhead[C]{}
\hrule \medskip % Upper rule
\begin{minipage}{0.35\textwidth} 
\raggedright
\footnotesize
Fausto David Hernández Jasso \hfill\\   
317000928 \hfill\\
fausto.david.hernandez.jasso@ciencias.unam.mx
\end{minipage}
\begin{minipage}{0.4\textwidth} 
\centering 
\large 
Cálculo Diferencial e Integral II\\ 
\normalsize 
Funciones logaritmo, exponencial y trigonométricas\\ 
\end{minipage}
\begin{minipage}{0.24\textwidth} 
\raggedleft
\today\hfill\\
\end{minipage}
\medskip\hrule 
\bigskip
\section{Ejercicios}
\subsection{Ejercicio 1}
\noindent
Derive cada una de las siguientes funciones
\begin{align*}
    f(x) &= e^{e^{e^{e^{x}}}} \\
    f(x) &= \log{\left(1 + \log{\left(1 + \log{\left(1 + e^{1 + e^{1 + x}}\right)}\right)}\right)} \\
    f(x) &= \left(\sin{x}\right)^{\sin{\left(\sin{x}\right)}} \\
    f(x) &= e^{\left(\int_{0}^{x} e^{-t^2} \ dt\right)} \\
    f(x) &= x^x
\end{align*}
\subsubsection*{Solución}
Veamos \(f(x) = e^{e^{e^{e^{x}}}}\).
\newline
Sea \(h(x) = e^{x}\), notemos lo siguiente:
\begin{align*}
    h_{1}\left(x\right) &= \left(h \circ h\right)(x) \\
                        &= h(h(x)) \\ 
                        &= e^{e^{x}}
\end{align*}
Así tenemos que
\begin{align*}
    h_{1}'(x) &= \left(\left(h \circ h\right)(x)\right)' \\
              &= h'(h(x))\cdot h'(x)
\end{align*}
Sabemos que \(h'(x) = e^{x}\), entonces 
\begin{align*}
    h_{1}'(x) &= h'(e^{x}) \cdot e^{x} \\
              &= e^{e^{x}} \cdot e^{x}
\end{align*}
Veamos que 
\begin{align*}
    h_{2}(x) &= \left(h \circ h_{1}\right)(x) \\
          &= h\left(h_{1}(x) \right) \\
          &= e^{e^{e^{x}}}
\end{align*}
Así tenemos que
\begin{align*}
    h_{2}'(x) &= \left(\left(h \circ h_{1}\right)(x)\right)' \\
              &= h'(h_{1}(x))\cdot h_{1}'(x)
\end{align*}
Sabemos que \(h'(x) = e^{x}\) y \(h_{1}'(x) = e^{e^{x}} \cdot e^{x}\), entonces 
\begin{align*}
    h_{2}'(x) &= h'(e^{e^{x}}) \cdot e^{e^{x}} \cdot e^{x} \\
              &= e^{e^{e^{x}}} \cdot e^{e^{x}} \cdot e^{x}
\end{align*}
Veamos que 
\begin{align*}
    h_{3}(x) &= \left(h \circ h_{2}\right)(x) \\
          &= h\left(h_{2}(x) \right) \\
          &= e^{e^{e^{e^{x}}}}
\end{align*}
Así tenemos que
\begin{align*}
    h_{3}'(x) &= \left(\left(h \circ h_{2}\right)(x)\right)' \\
              &= h'(h_{2}(x))\cdot h_{2}'(x) \\
\end{align*}
Sabemos que \(h'(x) = e^{x}\) y \(h_{1}'(x) = e^{e^{x}} \cdot e^{x}\), entonces 
\begin{align*}
    h_{3}'(x) &= h'(e^{e^{e^{x}}}) \cdot e^{e^{e^{x}}} \cdot e^{e^{x}} \cdot e^{x}  \\
              &= e^{e^{e^{e^{x}}}} \cdot e^{e^{e^{x}}} \cdot e^{e^{x}} \cdot e^{x}
\end{align*}
Por lo tanto 
\[
    \boxed{\boxed{\left(e^{e^{e^{e^{x}}}}\right)' = e^{e^{e^{e^{x}}}} \cdot e^{e^{e^{x}}} \cdot e^{e^{x}} \cdot e^{x}}}  
\]
Veamos \(f(x) = \log{\left(1 + \log{\left(1 + \log{\left(1 + e^{1 + e^{1 + x}}\right)}\right)}\right)}\)
\newline
Sea \(g(x) = e^{1 + x}\), notemos lo siguiente:
\begin{align*}
    g_{1}\left(x\right) &= \left(g \circ g\right)(x) \\
                        &= g(g(x)) \\ 
                        &= e^{1 + e^{1 + x}}
\end{align*}
Así tenemos que
\begin{align*}
    g_{1}'(x) &= \left(\left(g \circ g\right)(x)\right)' \\
              &= g'(g(x))\cdot g'(x)
\end{align*}
Sabemos que \(g'(x) = e^{1 + x}\), entonces 
\begin{align*}
    g_{1}'(x) &= g'(e^{1 + x}) \cdot e^{1 + x} \\
              &= e^{e^{1 + x}} \cdot e^{1 + x}
\end{align*}
Sea \(h(x) = \log{\left(1 + x\right)}\), notemos lo siguiente:
\begin{align*}
    h_{1}\left(x\right) &= \left(h \circ g_{1}\right)(x) \\
                        &= h(g_{1}(x)) \\ 
                        &= \log{\left(1 + e^{1 + e^{1 + x}}\right)}  
\end{align*}
Así tenemos que
\begin{align*}
    h_{1}'(x) &= \left(\left(h \circ g_{1}\right)(x)\right)' \\
              &= h'(g_{1}(x))\cdot g_{1}'(x)
\end{align*}
Sabemos que \(h'(x) = \frac{1}{1 + x}\) y \(g_{1}'(x) = e^{e^{1 + x}} \cdot e^{1 + x} \), entonces
\begin{align*}
    h_{1}'(x) &= h'(e^{1 + e^{1 + x}}) \cdot e^{e^{1 + x}} \cdot e^{1 + x} \\
              &= \left(\frac{1}{1 + e^{1 + e^{1 + x}}}\right) \cdot e^{e^{1 + x}} \cdot e^{1 + x} \\
              &= \frac{e^{e^{1 + x}} \cdot e^{1 + x}}{1 + e^{1 + e^{1 + x}}}
\end{align*}
Veamos que 
\begin{align*}
    h_{2}\left(x\right) &= \left(h \circ h_{1}\right)(x) \\
                        &= h(h_{1}(x)) \\ 
                        &= \log{\left(1 + \log{\left(1 + e^{1 + e^{1 + x}}\right)}\right)}  
\end{align*}
Así tenemos que
\begin{align*}
    h_{2}'(x) &= \left(\left(h \circ h_{1}\right)(x)\right)' \\
              &= h'(h_{1}(x))\cdot h_{1}'(x)
\end{align*}
Sabemos que \(h'(x) = \frac{1}{1 + x}\) y \(h_{1}'(x) = \frac{e^{e^{1 + x}} \cdot e^{1 + x}}{1 + e^{1 + e^{1 + x}}} \), entonces
\begin{align*}
    h_{2}'(x) &= h'\left(\log{\left(1 + e^{1 + e^{1 + x}}\right)}\right) \cdot \frac{e^{e^{1 + x}} \cdot e^{1 + x}}{1 + e^{1 + e^{1 + x}}} \\
              &= \frac{1}{1 + \log{\left(1 + e^{1 + e^{1 + x}}\right)}} \cdot \frac{e^{e^{1 + x}} \cdot e^{1 + x}}{1 + e^{1 + e^{1 + x}}} \\
              &= \frac{e^{e^{1 + x}} \cdot e^{1 + x}}{\left(1 + \log{\left(1 + e^{1 + e^{1 + x}}\right)}\right)\left(1 + e^{1 + e^{1 + x}}\right)}
\end{align*}
Veamos que 
\begin{align*}
    h_{3}\left(x\right) &= \left(h \circ h_{2}\right)(x) \\
                        &= h(h_{2}(x)) \\ 
                        &= \log{\left(1 + \log{\left(1 + \log{\left(1 + e^{1 + e^{1 + x}}\right)}\right)}\right)}  
\end{align*}
Así tenemos que
\begin{align*}
    h_{3}'(x) &= \left(\left(h \circ h_{2}\right)(x)\right)' \\
              &= h'(h_{2}(x))\cdot h_{2}'(x)
\end{align*}
Sabemos que \(h'(x) = \frac{1}{1 + x}\) y \(h_{2}'(x) = \frac{e^{e^{1 + x}} \cdot e^{1 + x}}{\left(1 + \log{\left(1 + e^{1 + e^{1 + x}}\right)}\right)\left(1 + e^{1 + e^{1 + x}}\right)} \), entonces
\begin{align*}
    h_{3}'(x) &= h'\left(\log{\left(1 + \log{\left(1 + e^{1 + e^{1 + x}}\right)}\right)}\right) \cdot 
                    \frac{e^{e^{1 + x}} \cdot e^{1 + x}}{\left(1 + \log{\left(1 + e^{1 + e^{1 + x}}\right)}\right)\left(1 + e^{1 + e^{1 + x}}\right)} \\
              &= \frac{1}{1 + \log{\left(1 + \log{\left(1 + e^{1 + e^{1 + x}}\right)}\right)}} \cdot 
              \frac{e^{e^{1 + x}} \cdot e^{1 + x}}{\left(1 + \log{\left(1 + e^{1 + e^{1 + x}}\right)}\right)\left(1 + e^{1 + e^{1 + x}}\right)} \\ \\
              &= \frac{e^{e^{1 + x}} \cdot e^{1 + x}}{\left(1 + \log{\left(1 + \log{\left(1 + e^{1 + e^{1 + x}}\right)}\right)}\right)
              \left(1 + \log{\left(1 + e^{1 + e^{1 + x}}\right)}\right)\left(1 + e^{1 + e^{1 + x}}\right)}
\end{align*}
Por lo tanto 
\[
    \boxed{\boxed{\left(\log{\left(1 + \log{\left(1 + \log{\left(1 + e^{1 + e^{1 + x}}\right)}\right)}\right)}\right)'
    = \frac{e^{e^{1 + x}} \cdot e^{1 + x}}{\left(1 + \log{\left(1 + \log{\left(1 + e^{1 + e^{1 + x}}\right)}\right)}\right)
    \left(1 + \log{\left(1 + e^{1 + e^{1 + x}}\right)}\right)\left(1 + e^{1 + e^{1 + x}}\right)}}}
\]
Veamos \(f\left(x\right) = \left(\sin{x}\right)^{\sin{\left(\sin{x}\right)}}\)
\newline
Por las propiedades de la función exponencial y de la función logaritmo tenemos que
\begin{align*}
    f(x) &= \left(\sin{x}\right)^{\sin{\left(\sin{x}\right)}} \\
         &= e^{\sin{\left(\sin{x}\right)} \cdot \log{\left(\sin{x}\right)}} \\
\end{align*}
Así tenemos que 
\begin{align*}
    f'(x) &= \left(e^{\sin{\left(\sin{x}\right)} \cdot \log{\left(\sin{x}\right)}}\right)' \\
          &= e^{\sin{\left(\sin{x}\right)} \cdot \log{\left(\sin{x}\right)}} \cdot 
             \left(\sin{\left(\sin{x}\right)} \cdot \log{\left(\sin{x}\right)}\right)' \\
          &= e^{\sin{\left(\sin{x}\right)} \cdot \log{\left(\sin{x}\right)}} \cdot 
             \left(\left(\sin{\left(\sin{x}\right)}\right)'\cdot \log{\left(\sin{x}\right)} +
             \left(\log{\left(\sin{x}\right)}\right)' \cdot \sin{\left(\sin{x}\right)}
             \right) \\
          &= e^{\sin{\left(\sin{x}\right)} \cdot \log{\left(\sin{x}\right)}} \cdot 
             \left(\cos{\left(\sin{x}\right)}\cdot \cos{x}\cdot \log{\left(\sin{x}\right)} +
             \frac{\cos{x}}{\sin{x}} \cdot \sin{\left(\sin{x}\right)}
             \right) \\
          &= e^{\sin{\left(\sin{x}\right)} \cdot \log{\left(\sin{x}\right)}} \cdot 
             \left(\cos{\left(\sin{x}\right)}\cdot \cos{x}\cdot \log{\left(\sin{x}\right)} +
             \cot{x} \cdot \sin{\left(\sin{x}\right)}
             \right) \\
          &= \left(\sin{x}\right)^{\sin{\left(\sin{x}\right)}}\left(
            \cos{\left(\sin{x}\right)}\cdot \cos{x}\cdot \log{\left(\sin{x}\right)} +
            \cot{x} \cdot \sin{\left(\sin{x}\right)}
             \right)
\end{align*}
Por lo tanto
\[
    \boxed{\boxed{\left(\left(\sin{x}\right)^{\sin{\left(\sin{x}\right)}}\right)' = \left(\sin{x}\right)^{\sin{\left(\sin{x}\right)}}\left(
        \cos{\left(\sin{x}\right)}\cdot \cos{x}\cdot \log{\left(\sin{x}\right)} +
        \cot{x} \cdot \sin{\left(\sin{x}\right)}\right)}}
\]
Veamos \(f(x) = e^{\left(\int_{0}^{x} e^{-t^2} \ dt\right)} \\\)
\newline
Sea \(h(x) = e^{x}\) y \(g(x) = \int_{0}^{x} e^{-t^2} dt\), notemos lo siguiente:
\begin{align*}
    \left(h \circ g\right)(x) &= h(g(x)) \\
                              &= h\left(\int_{0}^{x} e^{-t^2} dt\right) \\
                              &= e^{\left(\int_{0}^{x} e^{-t^2} dt\right)} \\
\end{align*}
Así tenemos que
\begin{align*}
    f'(x) &= \left(h(g(x))\right)' \\
          &= h'(g(x))\cdot g'(x) \\
\end{align*}
Sabemos que \(h'(x) = e^{x}\) y \(g'(x) = e^{-x^2}\) ya que \(e^{-t^2}\) es continua en todo \(\mathbb{R}\)
y por el \textbf{Primer Teorema Fundamental tenemos que} \(\left(\int_{0}^{x} e^{-t^2} dt\right)' = e^{-x^2}\),
entonces
\begin{align*}
    f'(x) &= h'\left(\int_{0}^{x} e^{-t^2} dt\right)\cdot e^{-x^2} \\
          &= e^{\left(\int_{0}^{x} e^{-t^2} dt\right)}\cdot e^{-x^2}
\end{align*}
Por lo tanto 
\[
    \boxed{\boxed{\left(e^{\left(\int_{0}^{x} e^{-t^2} dt\right)}\right)' = e^{\left(\int_{0}^{x} e^{-t^2} dt\right)}\cdot e^{-x^2}}}  
\]
Veamos \(f(x) = x^{x}\)
Por las propiedades de la función exponencial y de la función logaritmo tenemos que
\begin{align*}
    f(x) &= \left(\sin{x}\right)^{\sin{\left(\sin{x}\right)}} \\
         &= e^{x \log{x}} \\
\end{align*}
Así tenemos que 
\begin{align*}
    f'(x) &= \left(e^{x \log{x}}\right)' \\
          &= e^{x \log{x}}\left(x \log{x}\right)' \\
          &= e^{x \log{x}}\left(x \cdot \left(\log{x}\right)' + \log{x} \cdot x'\right) \\
          &= e^{x \log{x}}\left(x \cdot \left(\frac{1}{x}\right) + \log{x} \cdot 1\right) \\
          &= e^{x \log{x}}\left(1 + \log{x}\right) \\
          &= x^{x}\left(1 + \log{x}\right) \\
\end{align*}
Por lo tanto 
\[
    \boxed{\boxed{\left(x^x\right)' = x^{x}\left(1 + \log{x}\right)}}
\]
\newpage
\subsection{Ejercicio 2}
\noindent
\begin{enumerate}
    \item Compruebe que \(f'(x) = f(x)\left(\log{\left(f(x)\right)}\right)'\), sí \(f > 0\)
    y derivable
    \item Halle \(f'(x)\) para cada una de las siguientes funciones
    \begin{align*}
        f(x) &= \left(1 + x\right)\left(1 + e^{x^2}\right) \\
        f(x) &= \frac{(3 - x)^{\frac{1}{3}}x^2}{(1 - x)(3 + x)^{2^{3}}} \\
        f(x) &= \frac{e^{x} - e^{-x}}{e^{2x}(1 + x^3)}
    \end{align*}
\end{enumerate}
\subsubsection*{Solución}
Veamos (1)
\newline 
Sabemos que \(f(x) > 0\) \(\forall x \in D_{f}\) entonces \(\log{f(x)}\) está bien definido, 
también tenemos por hipótesis que \(f'(x)\) existe, también sabemos que \(\log{x}\) es derivable 
así \(\log{\left(f(x)\right)}\) es derivable. Consecuentemente:
\begin{align*}
    \left(\log{\left(f(x)\right)}\right)' &= \frac{1}{f(x)} \cdot f'(x) \\
                                          &= \frac{f'(x)}{f(x)} \\
\end{align*}
Despejando \(f'(x)\) tenemos que 
\begin{align*}
    f'(x) &= \left(\log{\left(f(x)\right)}\right)' \cdot f(x)
\end{align*}
Veamos (2)
\newline
Veamos \(f(x) = \left(1 + x\right)\left(1 + e^{x^2}\right)\). 
\newline 
Obteniendo \(\log\) de ambos lados
\begin{align*}
    \log{\left(f(x)\right)} &= \log{\left(\left(1 + x\right)\left(1 + e^{x^2}\right)\right)} \\
                            &= \log{\left(1 + x\right)} +  \log{\left(1 + e^{x^2}\right)}
\end{align*}
Derivando de ambos lados
\begin{align*}
    \frac{f'(x)}{f(x)} &= \frac{1}{1 + x} + \frac{1}{1 + e^{x^{2}}} \cdot e^{x^2} \cdot 2x \\
                       &= \frac{1}{1 + x} + \frac{2x\cdot e^{x^2}}{1 + e^{x^{2}}}
\end{align*}
Los anterior es equivalente a 
\begin{align*}
    f'(x) &= f(x) \left(\frac{1}{1 + x} + \frac{2x\cdot e^{x^2}}{1 + e^{x^{2}}}\right)  \\
    f'(x) &= \left(1 + x\right)\left(1 + e^{x^2}\right) \left(\frac{1}{1 + x} + \frac{2x\cdot e^{x^2}}{1 + e^{x^{2}}}\right)  \\
\end{align*}
Veamos \(f(x) = \frac{(3 - x)^{\frac{1}{3}}x^2}{(1 - x)(3 + x)^{2^{3}}} = \frac{(3 - x)^{\frac{1}{3}}x^2}{(1 - x)(3 + x)^{8}}\)
\newline
Obteniendo \(\log\) de ambos lados
\begin{align*}
    \log{\left(f(x)\right)} &= \log{\left(\frac{(3 - x)^{\frac{1}{3}}x^2}{(1 - x)(3 + x)^{8}}\right)} \\
                            &= \log{\left((3 - x)^{\frac{1}{3}}x^2\right)} - \log{\left((1 - x)(3 + x)^{8}\right)} \\
                            &= \log{\left((3 - x)^{\frac{1}{3}}\right)} + \log{\left(x^2\right)} - \left(\log{(1 - x)} + \log{\left((3 + x)^{8}\right)}\right) \\
                            &= \frac{1}{3}\log{\left(3 - x\right)} + 2\log{\left(x\right)} - \left(\log{(1 - x)} + 8\log{\left(3 + x\right)}\right) \\
                            &= \frac{1}{3}\log{\left(3 - x\right)} + 2\log{\left(x\right)} - \log{(1 - x)} - 8\log{\left(3 + x\right)} \\
\end{align*}
Derivando de ambos lados
\begin{align*}
    \frac{f'(x)}{f(x)} &= \frac{1}{3}\left(-\frac{1}{3 - x}\right) + 2\left(\frac{1}{x}\right) - \left(- \frac{1}{1 - x}\right) - 8\left(\frac{1}{3 + x}\right) \\
                       &= -\frac{1}{3(3 - x)} + \frac{2}{x} + \frac{1}{1 - x} - \frac{8}{3 + x} \\
\end{align*}
Lo anterior es equivalente a
\begin{align*}
    f'(x) &= f(x) \left(-\frac{1}{3(3 - x)} + \frac{2}{x} + \frac{1}{1 - x} - \frac{8}{3 + x}\right) \\
    f'(x) &= \left(\frac{(3 - x)^{\frac{1}{3}}x^2}{(1 - x)(3 + x)^{8}}\right) \left(-\frac{1}{3(3 - x)} + \frac{2}{x} + \frac{1}{1 - x} - \frac{8}{3 + x}\right) \\
\end{align*}
Veamos \(f(x) = \frac{e^{x} - e^{-x}}{e^{2x}(1 + x^3)}\)
\newline
Obteniendo \(\log\) de ambos lados
\begin{align*}
    \log{\left(f(x)\right)} &= \log{\left(\frac{e^{x} - e^{-x}}{e^{2x}(1 + x^3)}\right)} \\
                            &= \log{\left(e^{x} - e^{-x}\right)} - \log{\left(e^{2x}(1 + x^3)\right)} \\
                            &= \log{\left(e^{x} - e^{-x}\right)} - \left(\log{\left(e^{2x}\right)} + \log{\left(1 + x^3\right)}\right) \\
                            &= \log{\left(e^{x} - e^{-x}\right)} - \log{\left(e^{2x}\right)} - \log{\left(1 + x^3\right)} \\
                            &= \log{\left(e^{x} - e^{-x}\right)} - 2x - \log{\left(1 + x^3\right)} \\
\end{align*}
Derivando de ambos lados
\begin{align*}
    \frac{f'(x)}{f(x)} &= \left(\frac{1}{e^{x} - e^{-x}}\right)\left(e^{x} + e^{-x}\right) - 2 - \left(\frac{1}{1 + x^3}\right)\left(3x^2\right) \\
                       &= \frac{e^{x} + e^{-x}}{e^{x} - e^{-x}} - 2 - \frac{3x^2}{1 + x^3} \\
\end{align*}
Lo anterior es equivalente a
\begin{align*}
    f'(x) &= f(x) \left(\frac{e^{x} + e^{-x}}{e^{x} - e^{-x}} - 2 - \frac{3x^2}{1 + x^3}\right) \\
    f'(x) &= \left(\frac{e^{x} - e^{-x}}{e^{2x}(1 + x^3)}\right)\left(\frac{e^{x} + e^{-x}}{e^{x} - e^{-x}} - 2 - \frac{3x^2}{1 + x^3}\right)
\end{align*}
\subsection{Ejercicio 3}
\noindent
Hallar los siguientes límites mediante la regla de L'Hôpital
\begin{align*}
    \lim_{x \to 0} &\frac{\sin{x} - x + \frac{x^3}{6}}{x^3} \\
    \lim_{x \to 0} &\frac{\sin{x} - x + \frac{x^3}{6}}{x^4} \\
    \lim_{x \to 0} &\frac{\cos{x} - 1 + \frac{x^2}{2}}{x^2} \\
\end{align*}
\subsubsection*{Solución}
Veamos \(\displaystyle \lim_{x \to 0} \frac{\sin{x} - x + \frac{x^3}{6}}{x^3}\)
\begin{align*}
    \lim_{x \to 0} \frac{\sin{x} - x + \frac{x^3}{6}}{x^3} &= \frac{\displaystyle{\lim_{x \to 0}} \left(\sin{x} - x + \frac{x^3}{6}\right)}{\displaystyle{\lim_{x \to 0}}x^3} \\
    &= \frac{\displaystyle{\lim_{x \to 0}} \sin{x} - \displaystyle{\lim_{x \to 0}} x + \displaystyle{\lim_{x \to 0}}\frac{x^3}{6}}{\displaystyle{\lim_{x \to 0}}x^3} \\
    &= \frac{0}{0}
\end{align*}
Entonces podemos aplicar L'Hôpital
\begin{align*}
    \lim_{x \to 0} \frac{\sin{x} - x + \frac{x^3}{6}}{x^3} &= \lim_{x \to 0} \frac{\cos{x} - 1 + \frac{x^2}{2}}{3x^2} \\
    &= \frac{\displaystyle{\lim_{x \to 0}} \left(\cos{x} - 1 + \frac{x^2}{2}\right)}{\displaystyle{\lim_{x \to 0}} 3x^2} \\
    &= \frac{\displaystyle{\lim_{x \to 0}} \cos{x} - \displaystyle{\lim_{x \to 0}} 1 + \displaystyle{\lim_{x \to 0}} \frac{x^2}{2}}{\displaystyle{\lim_{x \to 0}} 3x^2} \\
    &= \frac{1 - 1 + 0}{0} \\
    &= \frac{0}{0}
\end{align*}
Podemos aplicar L'Hôpital
\begin{align*}
    \lim_{x \to 0} \frac{\cos{x} - 1 + \frac{x^2}{2}}{3x^2} &= \lim_{x \to 0} \frac{-\sin{x} + x}{6x} \\
    &= \frac{\displaystyle{\lim_{x \to 0}} \left(-\sin{x} + x\right)}{\displaystyle{\lim_{x \to 0}} 6x} \\
    &= \frac{-\displaystyle{\lim_{x \to 0}} \sin{x} + \displaystyle{\lim_{x \to 0}} x}{\displaystyle{\lim_{x \to 0}} 6x} \\
    &= \frac{- 0 + 0}{\displaystyle{0}} \\
    &= \frac{0}{\displaystyle{0}}
\end{align*}
Podemos aplicar L'Hôpital
\begin{align*}
    \lim_{x \to 0} \frac{-\sin{x} + x}{6x} &= \lim_{x \to 0} \frac{-\cos{x} + 1}{6} \\
    &= \frac{-\cos{0} + 1}{6} \\
    &= \frac{-1 + 1}{6} \\
    &= \frac{0}{6} \\
    &= 0
\end{align*}
Por lo tanto
\[
    \boxed{\boxed{ \lim_{x \to 0} \frac{\sin{x} - x + \frac{x^3}{6}}{x^3} = 0 }}
\]
Veamos \(\displaystyle \lim_{x \to 0} \frac{\sin{x} - x + \frac{x^3}{6}}{x^4}\)
\begin{align*}
    \lim_{x \to 0} \frac{\sin{x} - x + \frac{x^3}{6}}{x^4} &= \frac{\displaystyle{\lim_{x \to 0}} \left(\sin{x} - x + \frac{x^3}{6}\right)}{\displaystyle{\lim_{x \to 0}}x^4} \\
    &= \frac{\displaystyle{\lim_{x \to 0}} \sin{x} - \displaystyle{\lim_{x \to 0}} x + \displaystyle{\lim_{x \to 0}}\frac{x^3}{6}}{\displaystyle{\lim_{x \to 0}}x^4} \\
    &= \frac{0}{0}
\end{align*}
Entonces podemos aplicar L'Hôpital
\begin{align*}
    \lim_{x \to 0} \frac{\sin{x} - x + \frac{x^3}{6}}{x^4} &= \lim_{x \to 0} \frac{\cos{x} - 1 + \frac{x^2}{2}}{4x^3} \\
    &= \frac{\displaystyle{\lim_{x \to 0}} \left(\cos{x} - 1 + \frac{x^2}{2}\right)}{\displaystyle{\lim_{x \to 0}} 4x^3} \\
    &= \frac{\displaystyle{\lim_{x \to 0}} \cos{x} - \displaystyle{\lim_{x \to 0}} 1 + \displaystyle{\lim_{x \to 0}} \frac{x^2}{2}}{\displaystyle{\lim_{x \to 0}} 4x^3} \\
    &= \frac{1 - 1 + 0}{0} \\
    &= \frac{0}{0}
\end{align*}
Podemos aplicar L'Hôpital
\begin{align*}
    \lim_{x \to 0} \frac{\cos{x} - 1 + \frac{x^2}{2}}{4x^3} &= \lim_{x \to 0} \frac{-\sin{x} + x}{12x^2} \\
    &= \frac{\displaystyle{\lim_{x \to 0}} \left(-\sin{x} + x\right)}{\displaystyle{\lim_{x \to 0}} 12x^2} \\
    &= \frac{-\displaystyle{\lim_{x \to 0}} \sin{x} + \displaystyle{\lim_{x \to 0}} x}{\displaystyle{\lim_{x \to 0}} 12x^2} \\
    &= \frac{- 0 + 0}{\displaystyle{0}} \\
    &= \frac{0}{\displaystyle{0}}
\end{align*}
Podemos aplicar L'Hôpital
\begin{align*}
    \lim_{x \to 0} \frac{-\sin{x} + x}{12x^2} &= \lim_{x \to 0} \frac{-\cos{x} + 1}{24x} \\
    &= \frac{\displaystyle{\lim_{x \to 0}}\left(-\cos{x} + 1\right)}{\displaystyle{\lim_{x \to 0}} 24x} \\
    &= \frac{ - \displaystyle{\lim_{x \to 0}} \cos{x} + \displaystyle{\lim_{x \to 0}} 1}{\displaystyle{\lim_{x \to 0}} 24x} \\
    &= \frac{ - 1 + 1}{0} \\
    &= \frac{0}{0}
\end{align*}
Podemos aplicar L'Hôpital
\begin{align*}
    \lim_{x \to 0} \frac{-\cos{x} + 1}{24x} &= \lim_{x \to 0} \frac{\sin{x}}{24} \\
    &= \frac{0}{24} \\
    &= 0
\end{align*}
Por lo tanto 
\[
    \boxed{\boxed{\lim_{x \to 0} \frac{\sin{x} - x + \frac{x^3}{6}}{x^4} = 0}}
\]
Veamos \(\displaystyle{\lim_{x \to 0}} \frac{\cos{x} - 1 + \frac{x^2}{2}}{x^2}\)
\begin{align*}
    \lim_{x \to 0} \frac{\cos{x} - 1 + \frac{x^2}{2}}{x^2} &= \frac{\displaystyle{\lim_{x \to 0}}  \left(\cos{x} - 1 + \frac{x^2}{2}\right)}{\displaystyle{\lim_{x \to 0}} x^2} \\
    &= \frac{\displaystyle{\lim_{x \to 0}}  \cos{x} - \displaystyle{\lim_{x \to 0}} 1 + \displaystyle{\lim_{x \to 0}} \frac{x^2}{2}}{\displaystyle{\lim_{x \to 0}} x^2} \\
    &= \frac{1 - 1 + 0}{0} \\
    &= \frac{0}{0}
\end{align*}
Podemos aplicar L'Hôpital
\begin{align*}
    \lim_{x \to 0} \frac{\cos{x} - 1 + \frac{x^2}{2}}{x^2} &= \lim_{x \to 0} \frac{-\sin{x} + x}{2x} \\
    &= \frac{\displaystyle{\lim_{x \to 0}} \left(-\sin{x} + x\right)}{\displaystyle{\lim_{x \to 0}}2x} \\
    &= \frac{-\displaystyle{\lim_{x \to 0}} \sin{x} + \displaystyle{\lim_{x \to 0}} x}{\displaystyle{\lim_{x \to 0}}2x} \\
    &= \frac{-0 + 0}{0} \\
    &= \frac{0}{0} \\
\end{align*}
Podemos aplicar L'Hôpital
\begin{align*}
    \lim_{x \to 0} \frac{-\sin{x} + x}{2x} &= \lim_{x \to 0} \frac{-\cos{x} + 1}{2} \\
    &= \lim_{x \to 0} \frac{-\cos{x} + 1}{2} \\
    &= \frac{-\cos{0} + 1}{2} \\
    &= \frac{-1 + 1}{2} \\
    &= \frac{0}{2} \\
    &= 0 \\
\end{align*}
\end{document}