\documentclass[a4paper]{article} 
\input{head}
\newcommand{\pow}[2]{#1^{#2}}
\newcommand{\supra}[1]{\textsuperscript{#1}}
\begin{document}

%-------------------------------
%	TITLE SECTION
%-------------------------------

\fancyhead[C]{}
\hrule \medskip % Upper rule
\begin{minipage}{0.35\textwidth} 
\raggedright
\footnotesize
Fausto David Hernández Jasso \hfill\\   
317000928 \hfill\\
fausto.david.hernandez.jasso@ciencias.unam.mx
\end{minipage}
\begin{minipage}{0.4\textwidth} 
\centering 
\large 
Cálculo Diferencial e Integral II\\ 
\normalsize 
Fundamentos de la Integral de Riemann\\ 
\end{minipage}
\begin{minipage}{0.24\textwidth} 
\raggedleft
\today\hfill\\
\end{minipage}
\medskip\hrule 
\bigskip
\section{Ejercicios}
\subsection{Ejercicio 1}
\noindent
Sea \(A \subset \mathbb{R}\) un conjunto no vacío acotado superiormente. Entonces
\[
    -\inf{\left(-A\right)} = \sup{\left(A\right)}
\]
\subsubsection*{Solución}
\begin{proof}
    Por definición sabemos que \(A \neq \varnothing\) y además existe un \(M \in \mathbb{R}\)
    tal que \(a \leq M\) para todo \(a \in A\). Sabemos que por definición de \textbf{supremo}
    se cumple que \(\sup{\left(A\right)} \leq M\). 
    También sabemos que
    \begin{align*}
        a &\leq M &\forall a \in A \\
        (-1) \cdot a &\geq (-1) \cdot M &\forall a \in A \\
        -a &\geq -M &\forall a \in A
    \end{align*}
    Recordemos la definición de \(-A\)
    \[
        -A = \{-a \ : \ a \in A\}
    \]
    De la definición de \(-A\), sabemos que \(-A \neq \varnothing\).
    Notemos que \(-M\) es una cota inferior del conjunto \(-A\)
    consecuentemente \(-A\) está cotado inferiormente. Sea \(\beta\) una cota 
    superior de \(A\) entonces por definición de \(\sup{\left(A\right)}\) tenemos que
    \begin{align*}
        \beta &\geq \sup{\left(A\right)} \geq a & \forall a \in A \\
        (-1) \cdot \beta &\leq (-1) \cdot \sup{\left(A\right)} \leq (-1) \cdot a & \forall a \in A \\
        -\beta &\leq -\sup{\left(A\right)} \leq -a & \forall a \in A
    \end{align*}
    Lo anterior pasa para cualquier cota superior \(\beta\) y por ende a cualquier cota inferior \(-\beta\).
    \newline
    Así notemos que \(-\sup{\left(A\right)}\) es una cota inferior de \(-A\) y además es la cota inferior 
    más grande, es decir
    \begin{align*}
        \inf{\left(-A\right)} &= -\sup{\left(A\right)} \\
        -\inf{\left(-A\right)} &= \sup{\left(A\right)}
    \end{align*}
\end{proof}
\end{document}