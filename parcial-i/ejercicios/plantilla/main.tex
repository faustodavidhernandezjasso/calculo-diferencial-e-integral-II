\documentclass[a4paper]{article} 
\input{head}
\newcommand{\pow}[2]{#1^{#2}}
\newcommand{\supra}[1]{\textsuperscript{#1}}
\begin{document}

%-------------------------------
%	TITLE SECTION
%-------------------------------

\fancyhead[C]{}
\hrule \medskip % Upper rule
\begin{minipage}{0.35\textwidth} 
\raggedright
\footnotesize
Fausto David Hernández Jasso \hfill\\   
317000928 \hfill\\
fausto.david.hernandez.jasso@ciencias.unam.mx
\end{minipage}
\begin{minipage}{0.4\textwidth} 
\centering 
\large 
Cálculo Diferencial e Integral II\\ 
\normalsize 
Fundamentos de la Integral de Riemann\\ 
\end{minipage}
\begin{minipage}{0.24\textwidth} 
\raggedleft
\today\hfill\\
\end{minipage}
\medskip\hrule 
\bigskip
\section{Ejercicios}
\subsection{Ejercicio 1}
\noindent
Sea \(A \subset \mathbb{R}\) un conjunto no vacío acotado superiormente. Entonces
\[
    -\inf{\left(-A\right)} = \sup{\left(A\right)}
\]
\subsubsection*{Solución}
\begin{proof}
    Por definición sabemos que \(A \neq \varnothing\) y además existe un \(M \in \mathbb{R}\)
    tal que \(a \leq M\) para todo \(a \in A\). Sabemos que por definición de \textbf{supremo}
    se cumple que \(\sup{\left(A\right)} \leq M\). 
    También sabemos que
    \begin{align*}
        a &\leq M &\forall a \in A \\
        (-1) \cdot a &\geq (-1) \cdot M &\forall a \in A \\
        -a &\geq -M &\forall a \in A
    \end{align*}
    Recordemos la definición de \(-A\)
    \[
        -A = \{-a \ : \ a \in A\}
    \]
    De la definición de \(-A\), sabemos que \(-A \neq \varnothing\).
    Notemos que \(-M\) es una cota inferior del conjunto \(-A\)
    consecuentemente \(-A\) está cotado inferiormente. Sea \(\beta\) una cota 
    superior de \(A\) entonces por definición de \(\sup{\left(A\right)}\) tenemos que
    \begin{align*}
        \beta &\geq \sup{\left(A\right)} \geq a & \forall a \in A \\
        (-1) \cdot \beta &\leq (-1) \cdot \sup{\left(A\right)} \leq (-1) \cdot a & \forall a \in A \\
        -\beta &\leq -\sup{\left(A\right)} \leq -a & \forall a \in A
    \end{align*}
    Lo anterior pasa para cualquier cota superior \(\beta\) y por ende para cualquier cota inferior \(-\beta\).
    \newline
    Así notemos que \(-\sup{\left(A\right)}\) es una cota inferior de \(-A\) y además es la cota inferior 
    más grande, es decir
    \begin{align*}
        \inf{\left(-A\right)} &= -\sup{\left(A\right)} \\
        -\inf{\left(-A\right)} &= \sup{\left(A\right)}
    \end{align*}
\end{proof}
\subsection{Ejercicio 2}
Suponga que \(f\) es creciente. Demuestre que
\[
    \int_{a}^{b} f^{-1} = bf^{-1}(b) - af^{-1}(a) - \int_{f^{-1}(a)}^{f^{-1}(b)}f
\]
\subsubsection*{Solución}
\noindent
Del problema sabemos lo siguiente 
\begin{itemize}
    \item \(f\) es biyectiva en \([a, b]\) ya que \(f^{-1}\) existe y es su inversa.
    \item Como \(f\) es creciente en \([a, b]\) entonces \(f^{-1}\) también es 
    creciente en \([f\left(a\right), f\left(b\right)]\).
    \item Como \(f\) es creciente en \([a, b]\) entonces \(f\) es integrable \([a, b]\).
    \item Como \(f^{-1}\) es creciente en \([f(a), f(b)]\) entonces \(f^{-1}\) es integrable \([f(a), f(b)]\).
\end{itemize}
\begin{proof}
    Sea \(P = \{t_{0}, \dotsc, t_{n}\}\) una partición del intervalo \([a, b]\) y sea 
\(P' = \{f^{-1}(t_{0}), \dotsc, f^{-1}(t_{n})\}\) una partición del intervalo \([f^{-1}\left(a\right), 
f^{-1}\left(b\right)]\). Mostraremos que 
\begin{align*}
    L\left(f^{-1}, P\right) + U\left(f, P'\right) &= bf^{-1}\left(b\right) - af^{-1}\left(a\right) \\
    U\left(f^{-1}, P\right) + L\left(f, P'\right) &= bf^{-1}\left(b\right) - af^{-1}\left(a\right)
\end{align*}
Seguiremos las siguientes definiciones
\begin{align*}
    m_{i} &= \inf{\left(\{f(x) \ | x \in [f^{-1}\left(t_{i - 1}\right), f^{-1}\left(t_{i}\right)] \ \}\right)} \\
    M_{i} &= \sup{\left(\{f(x) \ | x \in [f^{-1}\left(t_{i - 1}\right), f^{-1}\left(t_{i}\right)] \ \}\right)} \\
    m'_{i} &= \inf{\left(\{f^{-1}(x) \ | \ x \in [t_{i - 1}, t_{i}] \ \}\right)} \\
    M'_{i} &= \sup{\left(\{f^{-1}(x) \ | \ x \in [t_{i - 1}, t_{i}] \ \}\right)}
\end{align*}
Veamos \(L\left(f^{-1}, P\right) + U\left(f, P'\right) = bf^{-1}\left(b\right) - af^{-1}\left(a\right)\)
\newline 
\begin{align*}
    L\left(f^{-1}, P\right) &= \sum_{i = 1}^{n} m'_{i}(t_{i} - t_{i - 1})
\end{align*}
Como \(f^{-1}\) también es creciente tenemos que \(m'_{i} = f^{-1}\left(t_{i - 1}\right)\). 
consecuentemente
\begin{align*}
    L\left(f^{-1}, P\right) &= \sum_{i = 1}^{n} m'_{i}(t_{i} - t_{i - 1}) \\
                            &= \sum_{i = 1}^{n} f^{-1}\left(t_{i - 1}\right)(t_{i} - t_{i - 1})
\end{align*}
\begin{align*}
    U\left(f, P'\right) &= \sum_{i = 1}^{n} M_{i}(f^{-1}\left(t_{i}\right) - f^{-1}\left(t_{i - 1}\right))
\end{align*}
Como \(f\) es creciente entonces tenemos que \(M_{i} = f\left(f^{-1}\left(t_{i}\right)\right) = t_{i}\)
\begin{align*}
    U\left(f, P'\right) &= \sum_{i = 1}^{n} M_{i}(f^{-1}\left(t_{i}\right) - f^{-1}\left(t_{i - 1}\right)) \\
                        &= \sum_{i = 1}^{n} t_{i}(f^{-1}\left(t_{i}\right) - f^{-1}\left(t_{i - 1}\right))
\end{align*}
\begin{align*}
    L\left(f^{-1}, P\right) + U\left(f, P'\right) &= \sum_{i = 1}^{n} f^{-1}\left(t_{i - 1}\right)(t_{i} - t_{i - 1})
     + \sum_{i = 1}^{n} t_{i}(f^{-1}\left(t_{i}\right) - f^{-1}\left(t_{i - 1}\right)) \\
    &= \left(f^{-1}\left(t_{0}\right)t_{1} - f^{-1}\left(t_{0}\right)t_{0}\right) + 
       \left(f^{-1}\left(t_{1}\right)t_{2} - f^{-1}\left(t_{1}\right)t_{1}\right) + \dotsc + \left(f^{-1}\left(t_{n - 1}\right)t_{n} - f^{-1}\left(t_{n - 1}\right)t_{n - 1}\right) \\
    &\ + \left(f^{-1}\left(t_{1}\right)t_{1} - f^{-1}\left(t_{0}\right)t_{1}\right) 
       + \left(f^{-1}\left(t_{2}\right)t_{2} - f^{-1}\left(t_{1}\right)t_{2}\right) + \dotsc +
       \left(f^{-1}\left(t_{n}\right)t_{n} - f^{-1}\left(t_{n - 1}\right)t_{n}\right) \\
    &= \left(f^{-1}\left(t_{0}\right)t_{1} - f^{-1}\left(t_{0}\right)t_{1}\right) + 
       \left(f^{-1}\left(t_{1}\right)t_{1} - f^{-1}\left(t_{1}\right)t_{1}\right) +
       \left(f^{-1}\left(t_{1}\right)t_{2} - f^{-1}\left(t_{1}\right)t_{2}\right) \\
    &\ + \left(f^{-1}\left(t_{2}\right)t_{2} - f^{-1}\left(t_{2}\right)t_{2}\right) + \dotsc + 
       \left(f^{-1}\left(t_{n - 1}\right)t_{n} - f^{-1}\left(t_{n - 1}\right)t_{n}\right) \\
    &\ + \left(f^{-1}\left(t_{n - 1}\right)t_{n - 1} - f^{-1}\left(t_{n - 1}\right)t_{n - 1}\right) +
       \left(f^{-1}\left(t_{n}\right)t_{n} - f^{-1}\left(t_{0}\right)t_{0}\right) \\
    &= 0 + 0 + 0 + 0 + \dotsc + 0 + 0 + \left(f^{-1}\left(t_{n}\right)t_{n} - f^{-1}\left(t_{0}\right)t_{0}\right) \\
    &= f^{-1}\left(t_{n}\right)t_{n} - f^{-1}\left(t_{0}\right)t_{0} \\
\end{align*}
Recordando que \(t_0 = a\) y \(t_{n} = b\) entonces
\[
    f^{-1}\left(b\right)b - f^{-1}\left(a\right)a = bf^{-1}\left(b\right) - af^{-1}\left(a\right) 
\]
Así 
\[
    L\left(f^{-1}, P\right) + U\left(f, P'\right) = bf^{-1}\left(b\right) - af^{-1}\left(a\right) 
\]
\newpage
Veamos \(U\left(f^{-1}, P\right) + L\left(f, P'\right) = bf^{-1}\left(b\right) - af^{-1}\left(a\right)\)
\newline 
\begin{align*}
    U\left(f^{-1}, P\right) &= \sum_{i = 1}^{n} M'_{i}(t_{i} - t_{i - 1})
\end{align*}
Como \(f^{-1}\) también es creciente tenemos que \(M'_{i} = f^{-1}\left(t_{i}\right)\). 
consecuentemente
\begin{align*}
    U\left(f^{-1}, P\right) &= \sum_{i = 1}^{n} M'_{i}(t_{i} - t_{i - 1}) \\
                            &= \sum_{i = 1}^{n} f^{-1}\left(t_{i}\right)(t_{i} - t_{i - 1})
\end{align*}
\begin{align*}
    L\left(f, P'\right) &= \sum_{i = 1}^{n} m_{i}(f^{-1}\left(t_{i}\right) - f^{-1}\left(t_{i - 1}\right))
\end{align*}
Como \(f\) es creciente entonces tenemos que \(m_{i} = f\left(f^{-1}\left(t_{i - 1}\right)\right) = t_{i - 1}\)
\begin{align*}
    L\left(f, P'\right) &= \sum_{i = 1}^{n} m_{i}(f^{-1}\left(t_{i}\right) - f^{-1}\left(t_{i - 1}\right)) \\
                        &= \sum_{i = 1}^{n} t_{i - 1}(f^{-1}\left(t_{i}\right) - f^{-1}\left(t_{i - 1}\right))
\end{align*}
\begin{align*}
    U\left(f^{-1}, P\right) + L\left(f, P'\right) &= \sum_{i = 1}^{n} f^{-1}\left(t_{i}\right)(t_{i} - t_{i - 1})
                                                  + \sum_{i = 1}^{n} t_{i - 1}(f^{-1}\left(t_{i}\right) - f^{-1}\left(t_{i - 1}\right)) \\
                                                  &= (f^{-1}\left(t_{1}\right)t_{1} - f^{-1}\left(t_{1}\right)t_{0}) 
                                                  + \left(f^{-1}\left(t_{2}\right)t_{2} - f^{-1}\left(t_{2}\right)t_{1}\right) + 
                                                  \dotsc + \left(f^{-1}\left(t_{n}\right)t_{n} - f^{-1}\left(t_{n}\right)t_{n - 1}\right) \\
                                                  &\ + + \left(f^{-1}\left(t_{1}\right)t_{1} - f^{-1}\left(t_{0}\right)t_{1}\right) 
                                                  + \left(f^{-1}\left(t_{2}\right)t_{2} - f^{-1}\left(t_{1}\right)t_{2}\right) + \dotsc +
                                                  \left(f^{-1}\left(t_{n}\right)t_{n} - f^{-1}\left(t_{n - 1}\right)t_{n}\right) \\
                                               &= \left(f^{-1}\left(t_{0}\right)t_{1} - f^{-1}\left(t_{0}\right)t_{1}\right) + 
                                                  \left(f^{-1}\left(t_{1}\right)t_{1} - f^{-1}\left(t_{1}\right)t_{1}\right) +
                                                  \left(f^{-1}\left(t_{1}\right)t_{2} - f^{-1}\left(t_{1}\right)t_{2}\right) \\
                                               &\ + \left(f^{-1}\left(t_{2}\right)t_{2} - f^{-1}\left(t_{2}\right)t_{2}\right) + \dotsc + 
                                                  \left(f^{-1}\left(t_{n - 1}\right)t_{n} - f^{-1}\left(t_{n - 1}\right)t_{n}\right) \\
                                               &\ + \left(f^{-1}\left(t_{n - 1}\right)t_{n - 1} - f^{-1}\left(t_{n - 1}\right)t_{n - 1}\right) +
                                                  \left(f^{-1}\left(t_{n}\right)t_{n} - f^{-1}\left(t_{0}\right)t_{0}\right) \\
                                               &= 0 + 0 + 0 + 0 + \dotsc + 0 + 0 + \left(f^{-1}\left(t_{n}\right)t_{n} - f^{-1}\left(t_{0}\right)t_{0}\right) \\
                                               &= f^{-1}\left(t_{n}\right)t_{n} - f^{-1}\left(t_{0}\right)t_{0} \\
\end{align*}
Recordando que \(t_0 = a\) y \(t_{n} = b\) entonces
\[
    f^{-1}\left(b\right)b - f^{-1}\left(a\right)a = bf^{-1}\left(b\right) - af^{-1}\left(a\right) 
\]
Así 
\[
    U\left(f^{-1}, P\right) + L\left(f, P'\right) = bf^{-1}\left(b\right) - af^{-1}\left(a\right) 
\]
\begin{align*}
    L\left(f^{-1}, P\right) + U\left(f, P'\right) &= bf^{-1}\left(b\right) - af^{-1}\left(a\right) \\
    U\left(f^{-1}, P\right) + L\left(f, P'\right) &= bf^{-1}\left(b\right) - af^{-1}\left(a\right) \\
                                                  &\Rightarrow  \\
    L\left(f^{-1}, P\right) &= bf^{-1}\left(b\right) - af^{-1}\left(a\right) - U\left(f, P'\right) \\
    U\left(f^{-1}, P\right) &= bf^{-1}\left(b\right) - af^{-1}\left(a\right) - L\left(f, P'\right) \\
\end{align*}
Como \(f^{-1}\) es integrable entonces
\begin{align*}
    \int_{a}^{b} f &= \sup{\left(L\left(f^{-1}, P\right)\right)} \\
                   &= \sup{\left(bf^{-1}\left(b\right) - af^{-1}\left(a\right) - U\left(f, P'\right)\right)} \\
                   &= bf^{-1}\left(b\right) - af^{-1}\left(a\right) + \sup{\left(- U\left(f, P'\right)\right)} \\
                   &= bf^{-1}\left(b\right) - af^{-1}\left(a\right) - \inf{\left(U\left(f, P'\right)\right)} \\
                   &= bf^{-1}\left(b\right) - af^{-1}\left(a\right) - \int_{f^{-1}(a)}^{f^{-1}(b)} f \\
\end{align*}
\end{proof}
\subsection{Ejercicio 3}
La función de Tomae se define como
\[
    t(x) =
    \begin{cases}
        \frac{1}{n} & \text{si } x = \frac{m}{n} \\
        0 & \text{si } x \notin \mathbb{Q}
    \end{cases}
\]
Demuestra que \(t\) es integrable y calcula \(\int_{0}^{1} t\)
\subsubsection*{Solución}
\begin{proof}
    Sea \(\varepsilon > 0\) y \(N \in \mathbb{N}\) tal que \(\frac{1}{N + 1} < \frac{\epsilon}{2}\),
    definimos al conjutno \(A\) como sigue:
    \[
        A = \left\{\frac{1}{2}, \frac{1}{3}, \frac{2}{3}, \frac{1}{4}, \frac{3}{4}, \dotsc, \frac{1}{N}
        , \dotsc, \frac{N - 1}{N}\right\}
    \]
    es decir, 
    \[
        A = \left\{ x \in [0, 1] \ | \ x = \frac{p}{q} \ \text{y} \ \frac{1}{q} \geq \frac{\varepsilon}{2} \right\}
    \]
    Sí \(x \in A\) entonces \(t\left(x\right) \in \left\{\frac{1}{2}, \frac{1}{3}, \frac{1}{4}
    , \dotsc, \frac{1}{N}\right\}\), así sabemos que \(t\left(x\right) \leq 1\).
    \newline
    Sí \(x \notin A\) entonces \(t\left(x\right) \leq \frac{1}{N + 1} < \frac{\varepsilon}{2}\).
    \newline 
    Notemos que \(A\) es finito. Sea \(|A| = k\), y sea \(n > k\) tal que \(\frac{1}{n} < 
    \frac{\varepsilon}{4k}\). Sea \(P\) una partición del invervalo \([0, 1]\) definida como
    \[
        P = \{0 = t_{0}, t_{1}, \dotsc, t_{n - 1}, t_{n} = 1\}    
    \]
    y además cumple con que \(t_{i} - t_{i - 1} = \frac{1}{n}\) \(\forall i \in \{1, \dotsc, n\}\).
    \newline 
    Veamos \(L\left(t(x), P_{n}\right)\)
    \newline
    Por densidad de los números irracionales sabemos que
    \[
        [t_{i - 1}, t_{i}] \cap \mathbb{I} \neq \varnothing
    \]
    consecuentemente existe un \(x \in [t_{i-1}, t_{i}]\) tal que \(t\left(x\right) = 0\), por lo tanto 
    \(m_{i} = 0\) \(\forall i \in \{1, \dotsc, n\}\). Así
    \begin{align*}
        L\left(t(x), P\right) &= \sum_{i = 1}^{n} m_{i}(t_{i} - t_{i - 1}) \\
                              &= \sum_{i = 1}^{n} 0(t_{i} - t_{i - 1}) \\
                              &= \sum_{i = 1}^{n} 0 \\
                              &= 0 \\
    \end{align*}
    Veamos \(U\left(t(x), P_{n}\right)\)
    \newline 
    Por definición tenemos que
    \[
        U\left(t(x), P_{n}\right) = \sum_{i = 1}^{n} M_{i}\left(t_{i} - t_{i - 1}\right)
    \]
    Notemos que lo anterior es equivalente a
    \begin{align*}
        U\left(t(x), P_{n}\right) &= \sum_{i = 1}^{n} M_{i}\left(t_{i} - t_{i - 1}\right) \\
        &= \sum_{[t_{i - 1}, t_{i}] \cap A = \varnothing} M_{i}\left(t_{i} - t_{i - 1}\right)
          + \sum_{[t_{i - 1}, t_{i}] \cap A \neq \varnothing} M_{i}\left(t_{i} - t_{i - 1}\right) \\
    \end{align*}
    Sea \(i \in \{1, \dotsc, n\}\) tal que \([t_{i - 1}, t_{i}] \cap A \neq \varnothing\)
    entonces existe \(x \in [t_{i - 1}, t_{i}]\) tal que \(t\left(x\right) \in \left\{\frac{1}{2}, 
    \frac{1}{3}, \dotsc, \frac{1}{N}\right\}\), así \(t(x) \leq 1\) \(\forall x \in [t_{i - 1}, t_{i}]\)
    consecuentemente \(M_{i} \leq 1\). Hacemos la observación de que \(A\) interseca a lo más a \(2k\)
    intervalos de \(P\), recordemos que \(|A| = k\), entonces se puede dar el caso en el que cada 
    elemento de \(A\) sea igual a \(t_{i}\) para \(k\) elementos de \(P\), así 
    ese elemento de \(A\) está en el intervalo \([t_{i - 1}, t_{i}]\) y \([t_{i}, t_{i + 1}]\).
    Así 
    \begin{align*}
        \sum_{[t_{i - 1}, t_{i}] \cap A \neq \varnothing} M_{i}\left(t_{i} - t_{i - 1}\right) &\leq  
        \sum_{[t_{i - 1}, t_{i}] \cap A \neq \varnothing} 1\left(t_{i} - t_{i - 1}\right) \\
        &< \sum_{[t_{i - 1}, t_{i}] \cap A \neq \varnothing} \frac{\epsilon}{4k} \\
        &= \frac{\epsilon}{4k} \sum_{[t_{i - 1}, t_{i}] \cap A \neq \varnothing} 1 \\
        &\leq \frac{\epsilon}{4k} \sum_{j = 1}^{2k} 1 \\
        &= \frac{\epsilon}{4k}(2k) \\
        &= \frac{\epsilon}{2} \\
    \end{align*}
    Por lo tanto
    \[
        \sum_{[t_{i - 1}, t_{i}] \cap A \neq \varnothing} M_{i}\left(t_{i} - t_{i - 1}\right) < \frac{\epsilon}{2}
    \]
    Así tenemos que
    \[
        U\left(t(x), P\right) - L\left(t(x), P\right) = U\left(t(x), P\right) - 0 = U\left(t(x), P\right) - L\left(f, P\right) = U\left(f, P\right) < \frac{\varepsilon}{2} < \varepsilon 
    \]
    Por el \textbf{Criterio de Integrabilidad} tenemos que \(t\) es integrable.
\end{proof}
Calcularemos \(\int_{0}^{1}t\)
\newline 
Sea \(\varepsilon > 0\), por lo anterior sabemos que existe \(P\) una partición del intervalo 
\([0, 1]\) tal que 
\[
    U\left(t(x), P\right) - L\left(t(x), P\right) < \varepsilon
\]
Sabemos que \(\int_{0}^{1} t\) cumple lo siguiente 
\[
    L\left(t(x), P\right) \leq \int_{0}^{1} t \leq U\left(t(x), P\right)
\]
Pero ya hemos calculado tanto \(L\left(t(x), P\right)\) y \(U\left(t(x), P\right)\)
\[
    0 = L\left(t(x), P\right) \leq \int_{0}^{1} t \leq U\left(t(x), P\right) < \varepsilon
\]
Como lo anterior pasa para cualquier \(\varepsilon > 0\) entonces 
\[
    \int_{0}^{1} t = 0
\]
\end{document}