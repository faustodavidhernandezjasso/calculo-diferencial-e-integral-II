\documentclass[a4paper]{article} 
\addtolength{\hoffset}{-2.25cm}
\addtolength{\textwidth}{4.5cm}
\addtolength{\voffset}{-3.25cm}
\addtolength{\textheight}{5cm}
\setlength{\parskip}{0pt}
\setlength{\parindent}{0in}

%----------------------------------------------------------------------------------------
%	PACKAGES AND OTHER DOCUMENT CONFIGURATIONS
%----------------------------------------------------------------------------------------
\usepackage{mathtools} % Package for using math tools
\usepackage[inline]{enumitem} % Enumerate environment
\usepackage{blindtext} % Package to generate dummy text
\usepackage{charter} % Use the Charter font
\usepackage[utf8]{inputenc} % Use UTF-8 encoding
\usepackage{microtype} % Slightly tweak font spacing for aesthetics
\usepackage[spanish]{babel} % Language hyphenation and typographical rules
\usepackage{amsthm, amsmath, amssymb} % Mathematical typesetting
\usepackage{float} % Improved interface for floating objects
\usepackage[final, colorlinks = true, 
            linkcolor = black, 
            citecolor = black]{hyperref} % For hyperlinks in the PDF
\usepackage{graphicx, multicol} % Enhanced support for graphics
\usepackage{xcolor} % Driver-independent color extensions
\usepackage{marvosym, wasysym} % More symbols
\usepackage{rotating} % Rotation tools
\usepackage{censor} % Facilities for controlling restricted text
\usepackage{listings, style/lstlisting} % Environment for non-formatted code, !uses style file!
\usepackage{pseudocode} % Environment for specifying algorithms in a natural way
\usepackage{style/avm} % Environment for f-structures, !uses style file!
\usepackage{booktabs} % Enhances quality of tables
\usepackage{tikz-qtree} % Easy tree drawing tool
\tikzset{every tree node/.style={align=center,anchor=north},
         level distance=2cm} % Configuration for q-trees
\usepackage{style/btree} % Configuration for b-trees and b+-trees, !uses style file!
\usepackage[backend=biber,style=numeric,
            sorting=nyt]{biblatex} % Complete reimplementation of bibliographic facilities
\addbibresource{ecl.bib}
\usepackage{csquotes} % Context sensitive quotation facilities
\usepackage[yyyymmdd]{datetime} % Uses YEAR-MONTH-DAY format for dates
\renewcommand{\dateseparator}{-} % Sets dateseparator to '-'
\usepackage{fancyhdr} % Headers and footers
\usepackage{physics}
\pagestyle{fancy} % All pages have headers and footers
\fancyhead{}\renewcommand{\headrulewidth}{0pt} % Blank out the default header
\fancyfoot[L]{} % Custom footer text
\fancyfoot[C]{} % Custom footer text
\fancyfoot[R]{\thepage} % Custom footer text
\newcommand{\note}[1]{\marginpar{\scriptsize \textcolor{red}{#1}}} % Enables comments in red on margin


%----------------------------------------------------------------------------------------

\newcommand{\pow}[2]{#1^{#2}}
\newcommand{\supra}[1]{\textsuperscript{#1}}
\begin{document}

%-------------------------------
%	TITLE SECTION
%-------------------------------

\fancyhead[C]{}
\hrule \medskip % Upper rule
\begin{minipage}{0.35\textwidth} 
\raggedright
\footnotesize
Fausto David Hernández Jasso \hfill\\   
317000928 \hfill\\
fausto.david.hernandez.jasso@ciencias.unam.mx
\end{minipage}
\begin{minipage}{0.4\textwidth} 
\centering 
\large 
Cálculo Diferencial e Integral II\\ 
\normalsize 
Fundamentos de la Integral de Riemann\\ 
\end{minipage}
\begin{minipage}{0.24\textwidth} 
\raggedleft
\today\hfill\\
\end{minipage}
\medskip\hrule 
\bigskip
\section{Ejercicios}
\subsection{Ejercicio 1}
\noindent
Sea \(A \subset \mathbb{R}\) un conjunto no vacío acotado superiormente. Entonces
\[
    -\inf{\left(-A\right)} = \sup{\left(A\right)}
\]
\subsubsection*{Solución}
\begin{proof}
    Por definición sabemos que \(A \neq \varnothing\) y además existe un \(M \in \mathbb{R}\)
    tal que \(a \leq M\) para todo \(a \in A\). Sabemos que por definición de \textbf{supremo}
    se cumple que \(\sup{\left(A\right)} \leq M\). 
    También sabemos que
    \begin{align*}
        a &\leq M &\forall a \in A \\
        (-1) \cdot a &\geq (-1) \cdot M &\forall a \in A \\
        -a &\geq -M &\forall a \in A
    \end{align*}
    Recordemos la definición de \(-A\)
    \[
        -A = \{-a \ : \ a \in A\}
    \]
    De la definición de \(-A\), sabemos que \(-A \neq \varnothing\).
    Notemos que \(-M\) es una cota inferior del conjunto \(-A\)
    consecuentemente \(-A\) está cotado inferiormente. Sea \(\beta\) una cota 
    superior de \(A\) entonces por definición de \(\sup{\left(A\right)}\) tenemos que
    \begin{align*}
        \beta &\geq \sup{\left(A\right)} \geq a & \forall a \in A \\
        (-1) \cdot \beta &\leq (-1) \cdot \sup{\left(A\right)} \leq (-1) \cdot a & \forall a \in A \\
        -\beta &\leq -\sup{\left(A\right)} \leq -a & \forall a \in A
    \end{align*}
    Lo anterior pasa para cualquier cota superior \(\beta\) y por ende para cualquier cota inferior \(-\beta\).
    \newline
    Así notemos que \(-\sup{\left(A\right)}\) es una cota inferior de \(-A\) y además es la cota inferior 
    más grande, es decir
    \begin{align*}
        \inf{\left(-A\right)} &= -\sup{\left(A\right)} \\
        -\inf{\left(-A\right)} &= \sup{\left(A\right)}
    \end{align*}
\end{proof}
\subsection{Ejercicio 2}
Suponga que \(f\) es creciente. Demuestre que
\[
    \int_{a}^{b} f^{-1} = bf^{-1}(b) - af^{-1}(a) - \int_{f^{-1}(a)}^{f^{-1}(b)}f
\]
\subsubsection*{Solución}
\noindent
Del problema sabemos lo siguiente 
\begin{itemize}
    \item \(f\) es biyectiva en \([a, b]\) ya que \(f^{-1}\) existe y es su inversa.
    \item Como \(f\) es creciente en \([a, b]\) entonces \(f^{-1}\) también es 
    creciente en \([f\left(a\right), f\left(b\right)]\).
    \item Como \(f\) es creciente en \([a, b]\) entonces \(f\) es integrable \([a, b]\).
    \item Como \(f^{-1}\) es creciente en \([f(a), f(b)]\) entonces \(f^{-1}\) es integrable \([f(a), f(b)]\).
\end{itemize}
\begin{proof}
    Sea \(P = \{t_{0}, \dotsc, t_{n}\}\) una partición del intervalo \([a, b]\) y sea 
\(P' = \{f^{-1}(t_{0}), \dotsc, f^{-1}(t_{n})\}\) una partición del intervalo \([f^{-1}\left(a\right), 
f^{-1}\left(b\right)]\). Mostraremos que 
\begin{align*}
    L\left(f^{-1}, P\right) + U\left(f, P'\right) &= bf^{-1}\left(b\right) - af^{-1}\left(a\right) \\
    U\left(f^{-1}, P\right) + L\left(f, P'\right) &= bf^{-1}\left(b\right) - af^{-1}\left(a\right)
\end{align*}
Seguiremos las siguientes definiciones
\begin{align*}
    m_{i} &= \inf{\left(\{f(x) \ | x \in [f^{-1}\left(t_{i - 1}\right), f^{-1}\left(t_{i}\right)] \ \}\right)} \\
    M_{i} &= \sup{\left(\{f(x) \ | x \in [f^{-1}\left(t_{i - 1}\right), f^{-1}\left(t_{i}\right)] \ \}\right)} \\
    m'_{i} &= \inf{\left(\{f^{-1}(x) \ | \ x \in [t_{i - 1}, t_{i}] \ \}\right)} \\
    M'_{i} &= \sup{\left(\{f^{-1}(x) \ | \ x \in [t_{i - 1}, t_{i}] \ \}\right)}
\end{align*}
Veamos \(L\left(f^{-1}, P\right) + U\left(f, P'\right) = bf^{-1}\left(b\right) - af^{-1}\left(a\right)\)
\newline 
\begin{align*}
    L\left(f^{-1}, P\right) &= \sum_{i = 1}^{n} m'_{i}(t_{i} - t_{i - 1})
\end{align*}
Como \(f^{-1}\) también es creciente tenemos que \(m'_{i} = f^{-1}\left(t_{i - 1}\right)\). 
consecuentemente
\begin{align*}
    L\left(f^{-1}, P\right) &= \sum_{i = 1}^{n} m'_{i}(t_{i} - t_{i - 1}) \\
                            &= \sum_{i = 1}^{n} f^{-1}\left(t_{i - 1}\right)(t_{i} - t_{i - 1})
\end{align*}
\begin{align*}
    U\left(f, P'\right) &= \sum_{i = 1}^{n} M_{i}(f^{-1}\left(t_{i}\right) - f^{-1}\left(t_{i - 1}\right))
\end{align*}
Como \(f\) es creciente entonces tenemos que \(M_{i} = f\left(f^{-1}\left(t_{i}\right)\right) = t_{i}\)
\begin{align*}
    U\left(f, P'\right) &= \sum_{i = 1}^{n} M_{i}(f^{-1}\left(t_{i}\right) - f^{-1}\left(t_{i - 1}\right)) \\
                        &= \sum_{i = 1}^{n} t_{i}(f^{-1}\left(t_{i}\right) - f^{-1}\left(t_{i - 1}\right))
\end{align*}
\begin{align*}
    L\left(f^{-1}, P\right) + U\left(f, P'\right) &= \sum_{i = 1}^{n} f^{-1}\left(t_{i - 1}\right)(t_{i} - t_{i - 1})
     + \sum_{i = 1}^{n} t_{i}(f^{-1}\left(t_{i}\right) - f^{-1}\left(t_{i - 1}\right)) \\
    &= \left(f^{-1}\left(t_{0}\right)t_{1} - f^{-1}\left(t_{0}\right)t_{0}\right) + 
       \left(f^{-1}\left(t_{1}\right)t_{2} - f^{-1}\left(t_{1}\right)t_{1}\right) + \dotsc + \left(f^{-1}\left(t_{n - 1}\right)t_{n} - f^{-1}\left(t_{n - 1}\right)t_{n - 1}\right) \\
    &\ + \left(f^{-1}\left(t_{1}\right)t_{1} - f^{-1}\left(t_{0}\right)t_{1}\right) 
       + \left(f^{-1}\left(t_{2}\right)t_{2} - f^{-1}\left(t_{1}\right)t_{2}\right) + \dotsc +
       \left(f^{-1}\left(t_{n}\right)t_{n} - f^{-1}\left(t_{n - 1}\right)t_{n}\right) \\
    &= \left(f^{-1}\left(t_{0}\right)t_{1} - f^{-1}\left(t_{0}\right)t_{1}\right) + 
       \left(f^{-1}\left(t_{1}\right)t_{1} - f^{-1}\left(t_{1}\right)t_{1}\right) +
       \left(f^{-1}\left(t_{1}\right)t_{2} - f^{-1}\left(t_{1}\right)t_{2}\right) \\
    &\ + \left(f^{-1}\left(t_{2}\right)t_{2} - f^{-1}\left(t_{2}\right)t_{2}\right) + \dotsc + 
       \left(f^{-1}\left(t_{n - 1}\right)t_{n} - f^{-1}\left(t_{n - 1}\right)t_{n}\right) \\
    &\ + \left(f^{-1}\left(t_{n - 1}\right)t_{n - 1} - f^{-1}\left(t_{n - 1}\right)t_{n - 1}\right) +
       \left(f^{-1}\left(t_{n}\right)t_{n} - f^{-1}\left(t_{0}\right)t_{0}\right) \\
    &= 0 + 0 + 0 + 0 + \dotsc + 0 + 0 + \left(f^{-1}\left(t_{n}\right)t_{n} - f^{-1}\left(t_{0}\right)t_{0}\right) \\
    &= f^{-1}\left(t_{n}\right)t_{n} - f^{-1}\left(t_{0}\right)t_{0} \\
\end{align*}
Recordando que \(t_0 = a\) y \(t_{n} = b\) entonces
\[
    f^{-1}\left(b\right)b - f^{-1}\left(a\right)a = bf^{-1}\left(b\right) - af^{-1}\left(a\right) 
\]
Así 
\[
    L\left(f^{-1}, P\right) + U\left(f, P'\right) = bf^{-1}\left(b\right) - af^{-1}\left(a\right) 
\]
\newpage
Veamos \(U\left(f^{-1}, P\right) + L\left(f, P'\right) = bf^{-1}\left(b\right) - af^{-1}\left(a\right)\)
\newline 
\begin{align*}
    U\left(f^{-1}, P\right) &= \sum_{i = 1}^{n} M'_{i}(t_{i} - t_{i - 1})
\end{align*}
Como \(f^{-1}\) también es creciente tenemos que \(M'_{i} = f^{-1}\left(t_{i}\right)\). 
consecuentemente
\begin{align*}
    U\left(f^{-1}, P\right) &= \sum_{i = 1}^{n} M'_{i}(t_{i} - t_{i - 1}) \\
                            &= \sum_{i = 1}^{n} f^{-1}\left(t_{i}\right)(t_{i} - t_{i - 1})
\end{align*}
\begin{align*}
    L\left(f, P'\right) &= \sum_{i = 1}^{n} m_{i}(f^{-1}\left(t_{i}\right) - f^{-1}\left(t_{i - 1}\right))
\end{align*}
Como \(f\) es creciente entonces tenemos que \(m_{i} = f\left(f^{-1}\left(t_{i - 1}\right)\right) = t_{i - 1}\)
\begin{align*}
    L\left(f, P'\right) &= \sum_{i = 1}^{n} m_{i}(f^{-1}\left(t_{i}\right) - f^{-1}\left(t_{i - 1}\right)) \\
                        &= \sum_{i = 1}^{n} t_{i - 1}(f^{-1}\left(t_{i}\right) - f^{-1}\left(t_{i - 1}\right))
\end{align*}
\begin{align*}
    U\left(f^{-1}, P\right) + L\left(f, P'\right) &= \sum_{i = 1}^{n} f^{-1}\left(t_{i}\right)(t_{i} - t_{i - 1})
                                                  + \sum_{i = 1}^{n} t_{i - 1}(f^{-1}\left(t_{i}\right) - f^{-1}\left(t_{i - 1}\right)) \\
                                                  &= (f^{-1}\left(t_{1}\right)t_{1} - f^{-1}\left(t_{1}\right)t_{0}) 
                                                  + \left(f^{-1}\left(t_{2}\right)t_{2} - f^{-1}\left(t_{2}\right)t_{1}\right) + 
                                                  \dotsc + \left(f^{-1}\left(t_{n}\right)t_{n} - f^{-1}\left(t_{n}\right)t_{n - 1}\right) \\
                                                  &\ + + \left(f^{-1}\left(t_{1}\right)t_{1} - f^{-1}\left(t_{0}\right)t_{1}\right) 
                                                  + \left(f^{-1}\left(t_{2}\right)t_{2} - f^{-1}\left(t_{1}\right)t_{2}\right) + \dotsc +
                                                  \left(f^{-1}\left(t_{n}\right)t_{n} - f^{-1}\left(t_{n - 1}\right)t_{n}\right) \\
                                               &= \left(f^{-1}\left(t_{0}\right)t_{1} - f^{-1}\left(t_{0}\right)t_{1}\right) + 
                                                  \left(f^{-1}\left(t_{1}\right)t_{1} - f^{-1}\left(t_{1}\right)t_{1}\right) +
                                                  \left(f^{-1}\left(t_{1}\right)t_{2} - f^{-1}\left(t_{1}\right)t_{2}\right) \\
                                               &\ + \left(f^{-1}\left(t_{2}\right)t_{2} - f^{-1}\left(t_{2}\right)t_{2}\right) + \dotsc + 
                                                  \left(f^{-1}\left(t_{n - 1}\right)t_{n} - f^{-1}\left(t_{n - 1}\right)t_{n}\right) \\
                                               &\ + \left(f^{-1}\left(t_{n - 1}\right)t_{n - 1} - f^{-1}\left(t_{n - 1}\right)t_{n - 1}\right) +
                                                  \left(f^{-1}\left(t_{n}\right)t_{n} - f^{-1}\left(t_{0}\right)t_{0}\right) \\
                                               &= 0 + 0 + 0 + 0 + \dotsc + 0 + 0 + \left(f^{-1}\left(t_{n}\right)t_{n} - f^{-1}\left(t_{0}\right)t_{0}\right) \\
                                               &= f^{-1}\left(t_{n}\right)t_{n} - f^{-1}\left(t_{0}\right)t_{0} \\
\end{align*}
Recordando que \(t_0 = a\) y \(t_{n} = b\) entonces
\[
    f^{-1}\left(b\right)b - f^{-1}\left(a\right)a = bf^{-1}\left(b\right) - af^{-1}\left(a\right) 
\]
Así 
\[
    U\left(f^{-1}, P\right) + L\left(f, P'\right) = bf^{-1}\left(b\right) - af^{-1}\left(a\right) 
\]
\begin{align*}
    L\left(f^{-1}, P\right) + U\left(f, P'\right) &= bf^{-1}\left(b\right) - af^{-1}\left(a\right) \\
    U\left(f^{-1}, P\right) + L\left(f, P'\right) &= bf^{-1}\left(b\right) - af^{-1}\left(a\right) \\
                                                  &\Rightarrow  \\
    L\left(f^{-1}, P\right) &= bf^{-1}\left(b\right) - af^{-1}\left(a\right) - U\left(f, P'\right) \\
    U\left(f^{-1}, P\right) &= bf^{-1}\left(b\right) - af^{-1}\left(a\right) - L\left(f, P'\right) \\
\end{align*}
Como \(f^{-1}\) es integrable entonces
\begin{align*}
    \int_{a}^{b} f &= \sup{\left(L\left(f^{-1}, P\right)\right)} \\
                   &= \sup{\left(bf^{-1}\left(b\right) - af^{-1}\left(a\right) - U\left(f, P'\right)\right)} \\
                   &= bf^{-1}\left(b\right) - af^{-1}\left(a\right) + \sup{\left(- U\left(f, P'\right)\right)} \\
                   &= bf^{-1}\left(b\right) - af^{-1}\left(a\right) - \inf{\left(U\left(f, P'\right)\right)} \\
                   &= bf^{-1}\left(b\right) - af^{-1}\left(a\right) - \int_{f^{-1}(a)}^{f^{-1}(b)} f \\
\end{align*}
\end{proof}
\end{document}