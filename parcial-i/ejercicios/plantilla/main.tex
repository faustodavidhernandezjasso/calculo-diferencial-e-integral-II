\documentclass[a4paper]{article} 
\addtolength{\hoffset}{-2.25cm}
\addtolength{\textwidth}{4.5cm}
\addtolength{\voffset}{-3.25cm}
\addtolength{\textheight}{5cm}
\setlength{\parskip}{0pt}
\setlength{\parindent}{0in}

%----------------------------------------------------------------------------------------
%	PACKAGES AND OTHER DOCUMENT CONFIGURATIONS
%----------------------------------------------------------------------------------------
\usepackage{mathtools} % Package for using math tools
\usepackage[inline]{enumitem} % Enumerate environment
\usepackage{blindtext} % Package to generate dummy text
\usepackage{charter} % Use the Charter font
\usepackage[utf8]{inputenc} % Use UTF-8 encoding
\usepackage{microtype} % Slightly tweak font spacing for aesthetics
\usepackage[spanish]{babel} % Language hyphenation and typographical rules
\usepackage{amsthm, amsmath, amssymb} % Mathematical typesetting
\usepackage{float} % Improved interface for floating objects
\usepackage[final, colorlinks = true, 
            linkcolor = black, 
            citecolor = black]{hyperref} % For hyperlinks in the PDF
\usepackage{graphicx, multicol} % Enhanced support for graphics
\usepackage{xcolor} % Driver-independent color extensions
\usepackage{marvosym, wasysym} % More symbols
\usepackage{rotating} % Rotation tools
\usepackage{censor} % Facilities for controlling restricted text
\usepackage{listings, style/lstlisting} % Environment for non-formatted code, !uses style file!
\usepackage{pseudocode} % Environment for specifying algorithms in a natural way
\usepackage{style/avm} % Environment for f-structures, !uses style file!
\usepackage{booktabs} % Enhances quality of tables
\usepackage{tikz-qtree} % Easy tree drawing tool
\tikzset{every tree node/.style={align=center,anchor=north},
         level distance=2cm} % Configuration for q-trees
\usepackage{style/btree} % Configuration for b-trees and b+-trees, !uses style file!
\usepackage[backend=biber,style=numeric,
            sorting=nyt]{biblatex} % Complete reimplementation of bibliographic facilities
\addbibresource{ecl.bib}
\usepackage{csquotes} % Context sensitive quotation facilities
\usepackage[yyyymmdd]{datetime} % Uses YEAR-MONTH-DAY format for dates
\renewcommand{\dateseparator}{-} % Sets dateseparator to '-'
\usepackage{fancyhdr} % Headers and footers
\usepackage{physics}
\pagestyle{fancy} % All pages have headers and footers
\fancyhead{}\renewcommand{\headrulewidth}{0pt} % Blank out the default header
\fancyfoot[L]{} % Custom footer text
\fancyfoot[C]{} % Custom footer text
\fancyfoot[R]{\thepage} % Custom footer text
\newcommand{\note}[1]{\marginpar{\scriptsize \textcolor{red}{#1}}} % Enables comments in red on margin


%----------------------------------------------------------------------------------------

\newcommand{\pow}[2]{#1^{#2}}
\newcommand{\supra}[1]{\textsuperscript{#1}}
\begin{document}

%-------------------------------
%	TITLE SECTION
%-------------------------------

\fancyhead[C]{}
\hrule \medskip % Upper rule
\begin{minipage}{0.35\textwidth} 
\raggedright
\footnotesize
Fausto David Hernández Jasso \hfill\\   
317000928 \hfill\\
fausto.david.hernandez.jasso@ciencias.unam.mx
\end{minipage}
\begin{minipage}{0.4\textwidth} 
\centering 
\large 
Cálculo Diferencial e Integral II\\ 
\normalsize 
Fundamentos de la Integral de Riemann\\ 
\end{minipage}
\begin{minipage}{0.24\textwidth} 
\raggedleft
\today\hfill\\
\end{minipage}
\medskip\hrule 
\bigskip
\section{Ejercicios}
\subsection{Ejercicio 1}
\noindent
Sea \(A \subset \mathbb{R}\) un conjunto no vacío acotado superiormente. Entonces
\[
    -\inf{\left(-A\right)} = \sup{\left(A\right)}
\]
\subsubsection*{Solución}
\begin{proof}
    Por definición sabemos que \(A \neq \varnothing\) y además existe un \(M \in \mathbb{R}\)
    tal que \(a \leq M\) para todo \(a \in A\). Sabemos que por definición de \textbf{supremo}
    se cumple que \(\sup{\left(A\right)} \leq M\). 
    También sabemos que
    \begin{align*}
        a &\leq M &\forall a \in A \\
        (-1) \cdot a &\geq (-1) \cdot M &\forall a \in A \\
        -a &\geq -M &\forall a \in A
    \end{align*}
    Recordemos la definición de \(-A\)
    \[
        -A = \{-a \ : \ a \in A\}
    \]
    De la definición de \(-A\), sabemos que \(-A \neq \varnothing\).
    Notemos que \(-M\) es una cota inferior del conjunto \(-A\)
    consecuentemente \(-A\) está cotado inferiormente. Sea \(\beta\) una cota 
    superior de \(A\) entonces por definición de \(\sup{\left(A\right)}\) tenemos que
    \begin{align*}
        \beta &\geq \sup{\left(A\right)} \geq a & \forall a \in A \\
        (-1) \cdot \beta &\leq (-1) \cdot \sup{\left(A\right)} \leq (-1) \cdot a & \forall a \in A \\
        -\beta &\leq -\sup{\left(A\right)} \leq -a & \forall a \in A
    \end{align*}
    Lo anterior pasa para cualquier cota superior \(\beta\) y por ende para cualquier cota inferior \(-\beta\).
    \newline
    Así notemos que \(-\sup{\left(A\right)}\) es una cota inferior de \(-A\) y además es la cota inferior 
    más grande, es decir
    \begin{align*}
        \inf{\left(-A\right)} &= -\sup{\left(A\right)} \\
        -\inf{\left(-A\right)} &= \sup{\left(A\right)}
    \end{align*}
\end{proof}
\end{document}